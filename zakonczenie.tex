\section{Zakończenie}

W~ramach pracy dyplomowej zaprojektowany i~wykonany został program
modelowania obiegów cieplnych \textit{plantcalc}. Za~jego pomocą
wykonany został bilans obiegu siłowni parowej
o~mocy~\SI{90}{\mega\watt}, a~otrzymane wyniki zostały porównane
z~wynikami otrzymanymi za~pomocą obliczeń oraz~profesjonalnego programu
Cycle-Tempo.

Wykonane obliczenia pozwoliły na~naprawienie wszystkich błędów
w~programie i~potwierdzenie jego poprawnej pracy przy~założonych
modelach. Wystąpiły różnice w~stosunku do~wyników otrzymanych
przy~pomocy Cycle-Tempo, przypisać je można jednak innemu sposobowi
modelowania upustów turbiny. Tym~samym uznać można, że~wykonana
aplikacja może~być stosowana do~wykonywania obliczeń obiegów cieplnych
w~zakresie zaimplementowanej funkcjonalności.

Najprawdopodobniej najważniejszym mankamentem programu jest brak wersji
graficznej. Zarówno schemat obiegu cieplnego, jak~i~zadania,
wprowadzane~są w~postaci kodu języka \Cpp. Wyniki najczęściej
wyprowadzane są w~postaci tabel bądź~grafów. Rozwiązanie takie
charakteryzuje~się znaczną elastycznością, dla~większości potencjalnych
użytkowników stanowić będzie jednak przeszkodę w~użytkowaniu programu.

Znanym błędem jest~również niedoskonałość algorytmu wykrywania
nieoznaczoności układu równań. Częstokroć niedostateczna ilość danych
skutkuje wynikami nieprawidłowymi bądź~wręcz absurdalnymi. Nie~powoduje
to~jednak problemów przy~pracy z~układami oznaczonymi.

Autor zamierza w~dalszym ciągu pracować nad~programem, a~w~szczególności
poprawić wyżej wymienione defekty. Do~innych możliwych udoskonaleń
zaliczyć można wprowadzenie dodatkowych modeli czynnika w~celu
modelowania organicznego obiegu Rankine'a czy~siłowni jądrowych.
W~dalszej perspektywie możliwe jest również wprowadzenie możliwości
modelowania obiegów gazowych.
