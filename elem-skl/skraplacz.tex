\subsubsection{Skraplacz}

\begin{figure}[H]
	\centering

	\subfloat[][Symbol]{
		\begin{tikzpicture}
			\node[condenser] (Sk) at (0,0) {Sk};

			{ [blue]
				\draw[stealth-] (Sk.n) -- ++(0,1.5)
					node[pos=.5, left] {$\dot m_i$}
					node[pos=.5, right] {$h_i$}
					node[pos=1] (tn) {};
				\draw[-stealth] (Sk.s) -- ++(0,-1.5)
					node[pos=.5, left] {$\dot m_o$}
					node[pos=.5, right] {$h_o$}
					node[pos=1] (ts) {};
				\draw[stealth-,dashed] (Sk.sw) -- ++(-.25,-.25)
					-- ++(-1.5,0)
					node[pos=.5, above] {$\dot m_i''$}
					node[pos=.5, below] {$h_i''$};
			}

			{ [green!50!black]
				\draw[-stealth] (tn) ++(1.75,0) |- (Sk.ne)
					node[pos=.225, left] {$\dot m_i'$}
					node[pos=.225, right] {$h_i'$};
				\draw[stealth-] (ts) ++(1.75,0) |- (Sk.se)
					node[pos=.225, left] {$\dot m_o'$}
					node[pos=.225, right] {$h_o'$};
			}
		\end{tikzpicture}
	}%
	~ %
	\wykresTs{
		\addplot+[no marks] table[x=s,y=T] {wykresy/skraplacz.txt}
			node[heatnode, pos=0] (st) {}
			node[pos=.5] (as) {}
			node[pos=.5005] (ae) {}
			node[heatnode, pos=1] (en) {};

		\node[right] at (st) {$\mathbf{S}_i$};
		\node[right] at (en) {$\mathbf{S}_o$};

		\draw[red, thick, -latex] (as.center) -- (ae.center);
	}

	\caption{Model skraplacza}
	\label{elem:skraplacz}
\end{figure}

\begin{description}

	\item[Parametry:] \hfill

		\begin{itemize}

			\item różnica temperatur na~wyprowadzeniach obiegu wtórnego
				$\Delta T$.

		\end{itemize}

	\item[Przemiany termodynamiczne:] \hfill

		\begin{itemize}

			\item $\mathbf{S_i} \rightarrow \mathbf{S_o}$,
				izobaryczne skraplanie czynnika;

			\item $\mathbf{S_i'} \rightarrow \mathbf{S_o'}$,
				izobaryczne ogrzewanie czynnika chłodzącego.

		\end{itemize}

	\item[Równania bilansowe:] \hfill

		\begin{itemize}

			\item \eq$\dot m_i + \dot m_i'' = \dot m_o$,

			\item \eq$\dot m_i' = \dot m_o'$,

			\item \eq$p_i = p_o$,

			\item \eq$p_i' = p_o'$,

			\item \eq$x_o = 0$,%
				\label{skraplacz-x0}

			\item \eq$T_o' = T_i' + \Delta T$,%
				\label{skraplacz-DT}

			\item \eq$\dot m_i h_i + \dot m_i' h_i' + \dot m_i'' h_i''
				= \dot m_o h_o + \dot m_o' h_o'$.

		\end{itemize}

\end{description}

Skraplacz modelowany jest jako~przeponowy wymiennik ciepła pracujący
w~dwóch obiegach. Model uwzględnia izobaryczną wymianę ciepła pomiędzy
czynnikami bez~ich mieszania. Równanie~\eqref{skraplacz-x0} dodatkowo
wymusza skroplenie czynnika roboczego (w~obiegu pierwotnym).

Dla~uproszczenia typowych obliczeń wprowadzona została dodatkowa
wielkość w~postaci różnicy temperatur czynnika na~początku i~na~końcu
obiegu chłodzącego. Przyjmuje ona domyślnie wartość \SI{10}{\kelvin}.

Przewidziano również wariant z~dodatkowym strumieniem powrotu skroplin
z~wymienników regeneracyjnych $S_i''$.
