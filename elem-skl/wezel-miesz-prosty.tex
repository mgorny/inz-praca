\subsubsection{Węzeł mieszający uproszczony}

\begin{figure}[H]
	\centering

	\begin{tikzpicture}
		\node[heatnode] (w) at (0,0) {};

		{ [red]
			\draw (w) -- +(-3,0)
				node[above, pos=.5] {$\dot m_i$}
				node[below, pos=.5] {$\mathrm{S}$};
			\draw[-stealth] (w) |- +(3,0)
				node[above, sloped, pos=.75] {$\dot m_o$}
				node[below, sloped, pos=.75] {$\mathrm{S}$};
		}

		\draw[stealth-, blue] (w.s) |- +(-3,-1.25)
			node[above, sloped, pos=.75] {$\dot m_i'$}
			node[below, sloped, pos=.75] {$\mathrm{S}'$};
	\end{tikzpicture}

	\caption{Schematyczne przedstawienie węzła mieszającego
		uproszczonego}
\end{figure}

Węzęł mieszający służy wprowadzaniu dodatkowego strumienia czynnika
do~strumienia głównego. W~węźle tym pomija~się mieszanie czynników;
zakłada~się, że~strumień wyjściowy ma~takie same parametry jak~strumień
wejściowy główny ($\mathrm{S}$).

Uproszczenie to ma na~celu umożliwienie budowy układu równań,
który~rozwiązany może~zostać bez~wykorzystania metod iteracyjnych.
Uwzględnienie mieszania czynnika spowodowałoby powstanie zależności
pomiędzy entalpią właściwą czynnika wyjściowego, a~strumieniami masy
dopływającymi do~węzła.

Uproszczenie takie jest dopuszczalne, jeżeli spełniony jest jeden
z~następujących warunków:

\begin{enumerate}

	\item parametry (entalpia właściwa) strumieni wejściowych są
		zbliżone co do~wartości: \hfill

		\hspace*{\fill} $\mathrm{S} \approx \mathrm{S}'$ \hspace*{\fill}

	\item strumień masy poboczny jest znacznie mniejszy niż~strumień
		masy główny: \hfill

		\hspace*{\fill} $\dot m_i' \ll \dot m_i$ \hspace*{\fill}

\end{enumerate}

Węzeł mieszający uproszczony dostarcza szereg równań, wyznaczających:

\begin{enumerate}

	\item bilans masowy węzła,

	\item równoważność parametrów czynnika na~wyjściu i~wejściu głównym
		węzła.

\end{enumerate}
