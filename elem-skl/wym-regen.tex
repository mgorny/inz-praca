\subsubsection{Podgrzewacz regeneracyjny wody}

\begin{figure}[H]
	\centering

	\subfloat[][Symbol]{
		\begin{tikzpicture}
			\node[heatxchg, font={\scriptsize}] (R) at (0,0) {R};

			{ [blue]
				\draw[stealth-] (R.n) -- ++(0,1.5)
					node[pos=.5, left] {$\dot m_i$}
					node[pos=.5, right] {$h_i$};
				\draw[-stealth] (R.se) |- ++(1.75,-1)
					node[pos=.775, above] {$\dot m_o$}
					node[pos=.775, below] {$h_o$};
				\draw[stealth-,dashed] (R.sw) |- ++(-1.75,-1)
					node[pos=.775, above] {$\dot m_i''$}
					node[pos=.775, below] {$h_i''$};
			}

			{ [green!50!black]
				\draw[stealth-] (R.e) -- ++(1.5,0)
					node[pos=.5, above] {$\dot m_i'$}
					node[pos=.5, below] {$h_i'$};
				\draw[-stealth] (R.w) -- ++(-1.5,0)
					node[pos=.5, above] {$\dot m_o'$}
					node[pos=.5, below] {$h_o'$};
			}
		\end{tikzpicture}
	}%
	~ %
	\wykresTs{
		\addplot+[no marks] table[x=s,y=T] {wykresy/podgrz.txt}
			node[heatnode, pos=0] (st) {}
			node[pos=.98] (as) {}
			node[pos=.9805] (ae) {}
			node[heatnode, pos=1] (en) {};
		\addplot+[no marks] table[x=s,y=T] {wykresy/podgrz-2.txt}
			node[heatnode, pos=0] (st2) {}
			node[pos=.8] (as2) {}
			node[pos=.9] (ae2) {}
			node[heatnode, pos=1] (en2) {};

		\node[right] at (st) {$\mathbf{S}_i$};
		\node[right] at (en) {$\mathbf{S}_o$};

		\draw[red, thick, -latex] (as.center) -- (ae.center);
		\draw[brown, thick, -latex] (as2.center) -- (ae2.center);
		\node[below] at (st2) {$\mathbf{S}_i'$};
	}

	\caption{Model podgrzewacza regeneracyjnego}
	\label{elem:podgrz}
\end{figure}

\begin{description}

	\item[Parametry:] \hfill

		\begin{itemize}

			\item różnica temperatur pomiędzy wyjściem obiegu
				pierwotnego, a~wtórnego $\Delta T$.

		\end{itemize}

	\item[Przemiany termodynamiczne:] \hfill

		\begin{itemize}

			\item $\mathbf{S_i} \rightarrow \mathbf{S_o}$,
				izobaryczne skraplanie czynnika;

			\item $\mathbf{S_i'} \rightarrow \mathbf{S_o'}$,
				izobaryczne ogrzewanie czynnika.

		\end{itemize}

	\item[Równania bilansowe:] \hfill

		\begin{itemize}

			\item \eq$\dot m_i + m_i'' = \dot m_o$,

			\item \eq$\dot m_i' = \dot m_o'$,

			\item \eq$p_i = p_o$,

			\item \eq$p_i' = p_o'$,

			\item \eq$x_o = 0$,%
				\label{regen-x0}

			\item \eq$T_o' = T_o - \Delta T$,%
				\label{regen-DT}

			\item \eq$\dot m_i h_i + \dot m_i' h_i' + \dot m_i'' h_i''
				= \dot m_o h_o + \dot m_o' h_o'$.

		\end{itemize}

\end{description}

Model wymiennika regeneracyjnego zbudowany jest w~sposób zbliżony
do~skraplacza. Co~istotne, podobnie jak~w~skraplaczu obieg skraplany
traktuje~się jako~obieg pierwotny; obieg podgrzewany natomiast stanowi
obieg wtórny.

Dla~uproszczenia typowych obliczeń wprowadzona została dodatkowa
wielkość w~postaci różnicy temperatur czynnika $\Delta T$ pomiędzy
skroplinami (temperatura wyższa), a~cieczą podgrzaną. Przyjmuje ona
domyślnie wartość \SI{5}{\kelvin}.
