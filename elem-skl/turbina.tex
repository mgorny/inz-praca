\subsubsection{Turbina (jednostkowa)}

\begin{figure}[H]
	\centering

	\subfloat[][Symbol]{
		\begin{tikzpicture}
			\node[turbine] (T) at (0,0) {T$_i$};

			{ [blue]
				\draw[stealth-] (T.nw) -- ++(0,1.5)
					node[pos=.5, left] {$\dot m_i$}
					node[pos=.5, right] {$h_i$};
				\draw[-stealth] (T.se) -- ++(0,-1.5)
					node[pos=.5, left] {$\dot m_o$}
					node[pos=.5, right] {$h_o$};
			}

			{ [blue, dashed, looseness=1, bend right=45, scale=.48, rotate=-45]
				\draw[-stealth] (T.ne) to ++(1,1) to ++(-1,1)
					to ++(-1,-1) to ++(1,-1);
					\node[above right] at (T.ne) {$h_o'$};
			}

			{ [red, dashed, double distance=.5mm]
				\draw[-stealth, double] (T.e) -- ++(2,0)
					node[pos=.5, above] {$P_m$};
			}
		\end{tikzpicture}
	}%
	~ %
	\wykresTs{
		\addplot+[no marks] table[x=s,y=T] {wykresy/turbina-i.txt};
		\addplot+[no marks, smooth] table[x=s,y=T] {wykresy/turbina.txt};
	}

	\caption{Model turbiny}
	\label{elem:turbina}
\end{figure}

\begin{description}

	\item[Parametry:] \hfill

		\begin{itemize}

			\item sprawność izentropowa turbiny $\eta_i$,

			\item sprawność mechaniczna turbiny $\eta_m$.

		\end{itemize}

	\item[Równania bilansowe:] \hfill

		\begin{itemize}

			\item \eq$\dot m_i = \dot m_o$,

			\item \eq$p_o' = p_o$,%
				\label{turb.petla.p}

			\item \eq$s_o' = s_i$,%
				\label{turb.petla.s}

			\item \eq$h_o = h_i + \eta_i ( h_o' - h_i )$,%
				\label{turb.wyj.h}

			\item \eq$\dot m_i h_i = \dot m_o h_o + \frac{P_m}{\eta_m}$.

		\end{itemize}

\end{description}

Model turbiny realizuje proces rozprężenia czynnika od~stanu
początkowego do~zadanego ciśnienia. Proces rozprężania dzieli~się
umyślnie na~dwa etapy:

\begin{enumerate}

	\item izentropowe rozprężanie czynnika do~zadanego ciśnienia $p_o$,

	\item wprowadzenie korekty ze~względu na~nieizentropowość
		rzeczywistego procesu rozprężania.

\end{enumerate}

Realizacji pierwszego z~wymienionych etapów dokonuje~się z~pomocą
tzw.~pętli izentropowej. Równania \eqref{turb.petla.p}
i~\eqref{turb.petla.s} wymuszają zamodelowanie stanu czynnika
po~izentropowym rozprężaniu $S_o'(p_o, s_o'=s_i)$.

Następnie, równanie~\eqref{turb.wyj.h} wykorzystywane jest
do~wyznaczenia entalpii czynnika po~rzeczywistym rozprężaniu $h_o$,
przy~uwzględnieniu zadanej sprawności izentropowej $\eta_i$. Wyznaczony
może zostać wyjściowy stan czynnika $S_o(p_o, h_o)$.

Model turbiny jednostkowej charakteryzuje~się pojedynczym wyjściem
czynnika. Realizacja turbiny upustowo-kondensacyjnej
oraz~upustowo-przeciwprężnej wymaga szeregowego połączenia elementów,
z~uwzględnieniem upustów w~postaci odgałęzień, oraz~węzła sumacyjnego
dla~wyprowadzeń mocy. Schematycznie układ taki zaprezentowany został
na~rys.~\ref{turb.upustowa}.

\begin{figure}[H]
	\centering

	\begin{tikzpicture}
		\node[turbine] (T1) at (0,0) {T$_1$};
		\node[turbine] (T2) at (4,0) {T$_2$};
		\node[turbine] (T3) at (8,0) {T$_3$};

		{ [blue]
			\draw[stealth-] (T1.nw) -- ++(0,1.5)
				node[pos=.5, left] {$\dot m_i$}
				node[pos=.5, right] {$h_i$};

			\draw[-stealth] (T1.se) -- ++(0,-1.5)
				node[pos=.5, left] {$\dot m_{x1}$}
				node[pos=.5, right] {$h_{x1}$}
				node[pos=.25, heatnode] (T1x) {};

			\draw[stealth-] (T2.nw) |- ++(-.1,.5) |- (T1x);

			\draw[-stealth] (T2.se) -- ++(0,-1.5)
				node[pos=.5, left] {$\dot m_{x2}$}
				node[pos=.5, right] {$h_{x2}$}
				node[pos=.25, heatnode] (T2x) {};

			\draw[stealth-] (T3.nw) |- ++(-.1,.5) |- (T2x);

			\draw[-stealth] (T3.se) -- ++(0,-1.5)
				node[pos=.5, left] {$\dot m_o$}
				node[pos=.5, right] {$h_o'$};
		}

		{ [red, double distance=.5mm]
			\node[sumnode] (S2) at (6.5,1.5) {};
			\node[sumnode] (S3) at (10.5,1.5) {};

			\draw[-stealth, double] (T1.e) -| (2.5,1.5)
				node[pos=0, above right] {$P_{m1}$}
				-- (S2.w);

			\draw[-stealth, double] (T2.e) -| (S2.s)
				node[pos=0, above right] {$P_{m2}$};

			\draw[-stealth, double] (S2.e) -- (S3.w);

			\draw[-stealth, double] (T3.e) -| (S3.s)
				node[pos=0, above right] {$P_{m3}$};

			\draw[-stealth, double] (S3.e) -- ++(1.5,0)
				node[pos=1, above left] {$P_m$};
		}
	\end{tikzpicture}

	\caption{Schematyczne przedstawienie połączeń dla~modelowania
		turbiny upustowo-kondensacyjnej}
	\label{turb.upustowa}
\end{figure}
