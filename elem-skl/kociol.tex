\subsubsection{Kocioł (oraz~przegrzew wtórny)}

\begin{figure}[H]
	\centering

	\subfloat[][Symbol kotła]{
		\begin{tikzpicture}
			\node[boiler] (K) at (0,0) {K};

			{ [blue]
				\draw[stealth-] (K.s) -- ++(0,-1.5)
					node[pos=.5, left] {$\dot m_i$}
					node[pos=.5, right] {$h_i$};
				\draw[-stealth] (K.n) -- ++(0,1.5)
					node[pos=.5, left] {$\dot m_o$}
					node[pos=.5, right] {$h_o$};
			}

			{ [red, dashed]
				\draw[stealth-] (K.w) -- ++(-2,0)
					node[pos=.5, above] {$\dot Q_f$};
			}
		\end{tikzpicture}
	}%
	~ %
	\subfloat[][Symbol przegrzewu wtórnego]{
		\begin{tikzpicture}
			\node[reheater] (K) at (2,0) {};
			\draw (K.s) -- ++(0,-1);

			{ [blue]
				\draw[stealth-] (K.s) ++(0,-1) -- ++(0,-1.5)
					node[pos=.5, left] {$\dot m_i$}
					node[pos=.5, right] {$h_i$};
				\draw[-stealth] (K.n) -- ++(0,1.5)
					node[pos=.5, left] {$\dot m_o$}
					node[pos=.5, right] {$h_o$};
			}

			{ [red, dashed]
				\draw[stealth-] (K.w) ++(0,-0.75) -- ++(-2,0)
					node[pos=.5, above] {$\dot Q_f$};
			}
		\end{tikzpicture}
	}%
	~ %
	\wykresTs{
		\addplot+[no marks] table[x=s,y=T] {wykresy/kociol.txt}
			node[heatnode, pos=0] (st) {}
			node[pos=.5923] (as) {}
			node[pos=.5924] (ae) {}
			node[heatnode, pos=1] (en) {};

		\node[right] at (st) {$\mathbf{S}_i$};
		\node[right] at (en) {$\mathbf{S}_o$};
		\draw[red, thick, -latex] (as.center) -- (ae.center);
	}

	\caption{Model kotła}
	\label{elem:kociol}
\end{figure}

\begin{description}

	\item[Parametry:] \hfill

		\begin{itemize}

			\item sprawność kotła $\eta_B$.

		\end{itemize}

	\item[Przemiany termodynamiczne:] \hfill

		\begin{itemize}

			\item $\mathbf{S_i} \rightarrow \mathbf{S_o}$,
				izobaryczne ogrzewanie, odparowanie i~przegrzanie
				czynnika.

		\end{itemize}

	\item[Równania bilansowe:] \hfill

		\begin{itemize}

			\item \eq$p_i = p_o$,

			\item \eq$\dot m_i = \dot m_o$,

			\item \eq$\dot m_i h_i + \eta_B \dot Q_f = \dot m_o h_o$.

		\end{itemize}

\end{description}

Kocioł zamodelowany jest w~postaci prostej maszyny przepływowej,
w~której energia zawarta w~paliwie wykorzystywana jest do~zwiększenia
entalpii czynnika. Model umożliwia uwzględnienie strat energii w~postaci
stałej sprawności.

Do~modelowania przegrzewu wtórnego wykorzystywany jest ten~sam obiekt.
Energia paliwa wykorzystana na~wtórne przegrzanie pary wyróżniana
jest~oddzielnie.

Tym~samym, modelowanie kotła wraz z~przegrzewem najczęściej wiąże~się
z~koniecznością rozgałęzienia strumienia paliwa oraz~założenia równych
sprawności. Przykładowe połączenie kotła z~przegrzewem wtórnym
przedstawiono na~rys.~\ref{kociol-z-przegrzewem}.

\begin{figure}[H]
	\centering

	\begin{tikzpicture}
			\node[boiler] (K) at (0,0) {K};
			\node[reheater] (Kr) at (2,0) {};
			\draw (Kr.s) -- ++(0,-1);

			{ [blue]
				\draw[stealth-] (K.s) -- ++(0,-1.5)
					node[pos=.5, left] {$\dot m_i$}
					node[pos=.5, right] {$h_i$};
				\draw[-stealth] (K.n) -- ++(0,1.5)
					node[pos=.5, left] {$\dot m_o$}
					node[pos=.5, right] {$h_o$};

				\draw[stealth-] (Kr.s) ++(0,-1) -- ++(0,-1.5)
					node[pos=.5, left] {$\dot m_i'$}
					node[pos=.5, right] {$h_i'$};
				\draw[-stealth] (Kr.n) -- ++(0,1.5)
					node[pos=.5, left] {$\dot m_o'$}
					node[pos=.5, right] {$h_o'$};
			}

			{ [red, dashed]
				\path (K.w) -- ++(-.5,1.25) node[heatnode] (Kp) {};

				\draw (Kp) -- ++(-1.5,0)
					node[pos=.5, above] {$\dot Q_f$};

				\draw[-stealth] (Kp) |- (K.w);

				\draw[stealth-] (Kr.w) ++(0,-.75) -| ++(-.5,1.25) -- (Kp);
			}
	\end{tikzpicture}

	\caption{Schematyczne przedstawienie połączeń dla~modelowania
		kotła z~przegrzewem wtórnym}
	\label{kociol-z-przegrzewem}
\end{figure}
