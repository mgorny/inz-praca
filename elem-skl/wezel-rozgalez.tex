\subsubsection{Węzeł rozgałęźny}

\begin{figure}[H]
	\centering

	\begin{tikzpicture}
		\node[heatnode] (w) at (0,0) {};

		{ [blue]
			\draw (w) -- +(-3,0)
				node[above, pos=.5] {$Q_i$}
				node[below, pos=.5] {$\mathrm{S}$};
			\draw[-stealth] (w) |- +(3,.75)
				node[above, sloped, pos=.75] {$Q_o$}
				node[below, sloped, pos=.75] {$\mathrm{S}$};
			\draw[-stealth] (w) |- +(3,-.75)
				node[above, sloped, pos=.75] {$Q_o'$}
				node[below, sloped, pos=.75] {$\mathrm{S}$};
		}
	\end{tikzpicture}

	\caption{Schematyczne przedstawienie węzła rozgałęźnego}
\end{figure}

Węzęł rozgałęźny służy wyprowadzaniu dwóch strumieni energii o~równych
parametrach z~jednego strumienia wejściowego. Węzły tego typu stosowane
są do~rozdziału czynnika, energii i~paliwa pomiędzy połączone równolegle
maszyny.

Węzeł rozgałęźny dostarcza szereg równań, wyznaczających:

\begin{enumerate}

	\item bilans energetyczny (masowy) węzła,

	\item równoważność parametrów po~wszystkich stronach węzła.

\end{enumerate}
