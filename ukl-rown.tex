\section{Modelowanie układu równań bilansowych}

\subsection{Kształt układu równań bilansowych}

Po~zakończeniu pierwszego etapu działań bilansowych otrzymuje~się układ
równań wraz z~szeregiem zadanych parametrów (wartości zmiennych). Układ
ten musi zostać zamodelowany cyfrowo i~rozwiązany dla~otrzymania
poszukiwanych parametrów.

Dla~właściwego doboru sposobu modelowania układu równań oraz~metod jego
rozwiązywania, należy w~pierwszej kolejności przedstawić ogólną
charakterystykę układu równań otrzymywanego wskutek bilansowania
zamodelowanej struktury obiegu.

\begin{figure}[h]
	\centering
	\begin{tikzpicture}[thick]
		\node[boiler] (K) at (0,3.5) {K};
		\node[turbine] (T1) at (2,4) {T$_1$};
		\node[turbine] (T2) at (5,4) {T$_2$};
		\node[condenser, font=\scriptsize] (Sk) at (5.75,2) {Sk};
		\node[pumpl, font=\scriptsize] (PS) at (4.5,.5) {PS};
		\node[heatxchg, font=\scriptsize] (R1) at (2.75,.5) {R1};

		\draw (Sk.s) |- (PS.e);
		\draw (PS.w) -- (R1.e);
		\draw (R1.w) -| (K.s);

		\path (Sk.sw) -- ++(-.25,-.25) node (SkW) {};
		\path (T1.nw) -- ++(0,.65) node[heatnode] (T1r) {};

		{ [-stealth]
			\draw (K.n) |- (T1r) -- (T1.nw);
			\draw (T1r.center) -| (T2.nw);
			\draw (T2.se) -- (Sk.n);

			\draw (T1.se) -| (R1.n);
			\draw (R1.s) -- ++(0,-.5) -| (SkW.center) -- (Sk.sw);
		}
	\end{tikzpicture}

	\caption{Przykładowy obieg dla~zilustrowania problemów rozwiązywania
	układu równań}
	\label{ukl-rown-przykl}
\end{figure}


Zamodelowanie przykładowego obiegu termodynamicznego, którego schemat
cieplny przedstawiony został na~rys.~\ref{ukl-rown-przykl} wymaga
rozwiązania układu ponad 150~równań liniowych oraz~wyznaczenia stanu
czynnika w~15~połączeniach (rurociągach). Wszystko to wiąże łącznie
ponad~200~zmiennych i~stałych.

Tak znaczne liczby związana są przede~wszystkim z~nadmiarowością
zmiennych. Zastosowanie prostego algorytmu upraszczającego równoważne
zmienne pozwoliłoby nie~tylko znacznie zmniejszyć liczbę zmiennych,
ale~również wyeliminować wiele równań, w~tym większość bilansów masy.

Jednakże, z~praktycznego punktu widzenia, rozwiązanie układu równań
tej~wielkości nie~stanowi problemu dla~współczesnych komputerów.
Tym~samym, eliminacja zmiennych stanowiłaby jedynie niepotrzebną
komplikację kodu programu.

Znacznie większy problem stanowi fakt, że~rozpatrywany układ częstokroć
jest~układem nieoznaczonym. Konieczne staje~się wyróżnienie
i~rozwiązanie oznaczonego podzbioru równań przy~jednoczesnym zachowaniu
pozostałych równań dla~ewentualnego późniejszego rozwiązania,
bądź~prawidłowe stwierdzenie sprzeczności.

Istotne staje~się również zagadnienie dokładności obliczeniowej.
Rozwiązanie układu równań częstokroć skutkuje otrzymaniem wielu
zbliżonych wartości dla~pojedynczej zmiennej. W~tej sytuacji konieczne
jest trafne stwierdzenie, czy~otrzymany wynik jest~poprawny, czy~też
układ równań jest~sprzeczny.


\subsection{Wybór sposobu modelowania układów równań}

Rozważone zostały dwie metody modelowania układów równań:

\begin{enumerate}

	\item modelowanie układu równań liniowych za~pomocą macierzy,

	\item modelowanie układu równań za~pomocą klas.

\end{enumerate}

Model macierzowy to~standardowy sposób zapisu układu równań w~programach
komputerowych. Charakteryzuje~się względną prostotą i~możliwością
bezpośredniego rozwiązywania przy~pomocy dostępnych bibliotek. Jednakże
model taki uwzględniać może jedynie równania liniowe.

Dlatego też programy korzystające z~modelu macierzowego zmuszone są
przeprowadzać dodatkowe obliczenia i~wyznaczać niezbędne parametry
czynnika termodynamicznego niezależnie od~procedur rozwiązywania układu
równań, najczęściej w~fazie wyznaczania go.

Model oparty na~klasach wykorzystuje możliwości programowania
obiektowego dla~opisu poszczególnych równań za~pomocą dedykowanych klas.
Model ten jest bardziej złożony od~macierzowego, jednakże umożliwia
modelowania dowolnych równań i~wykonywanie na~nich obliczeń.

W~programie zdecydowano~się wykorzystać model oparty na~klasach. W~ten
sposób wszystkie parametry obiegu przedstawione zostały przy~pomocy
zmiennych.


\subsection{Rodzaje równań}

\subsubsection{Równanie stanu czynnika}

Równania stanu czynnika stanowią szczególną grupę równań, określających
zależności pomiędzy właściwościami termodynamicznymi czynnika roboczego.
W~odniesieniu do~pary wodnej, równania te mają postać złożonych
wielomianów.

Ze~względów praktycznych, równania stanu czynnika nie~są rozpatrywane
przez~program w~pełnej postaci. Są one traktowane jako~abstrakcyjny
rodzaj równania, którego rozwiązanie zostaje zlecone zewnętrznej
bibliotece.

Równania stanu czynnika opisywane są w~programie za~pomocą funkcji:

\begin{itemize}
	\item $\mathbf{S}(p, T)$,
	\item $\mathbf{S}(p, h)$,
	\item $\mathbf{S}(p, s)$,
	\item $\mathbf{S}(h, s)$,
	\item $\mathbf{S}(p, x)$,
	\item $\mathbf{S}(T, x)$.
\end{itemize}


\subsubsection{Równanie liniowo-iloczynowe}

Równania liniowo-iloczynowe wykorzystywane są do~modelowania większości
spośród zależności występujących w~programie. Najogólniej mają one
postać:

\begin{equation}
	\sum_{k=1}^N C_k x_k y_k = 0
\end{equation}

$C_k$ stanowi dowolną stałą, natomiast $x_k$ i~$y_k$ zmienne (względnie
stałe).

Równanie tego typu może~być wykorzystane do~modelowania prostych równań
liniowych (bilansy masy), równań bilansów energii oraz~zależności
iloczynowych dwóch zmiennych.

W~praktyce najczęściej wartość przynajmniej jednej ze~zmiennych
(na~przykład entalpii właściwej) w~każdym członie jest znana. Wówczas
równanie tego typu sprowadzić można do~równania liniowego:

\begin{equation}
	\sum_{k=1}^N C_k C_k' x_k = 0
\end{equation}


\subsection{Metody rozwiązywania układów równań}

\subsubsection{Podział metod rozwiązywania układów równań}

Najogólniej metody rozwiązywania układów równań podzielić można na~dwie
grupy:

\begin{enumerate}

	\item metody nieiteracyjne (proste),

	\item metody iteracyjne.

\end{enumerate}

Metody proste charakteryzują~się tym, że~przy~określonym zestawie danych
wejściowych, każdorazowe wykonanie algorytmu skutkuje uzyskaniem jednego
i~tego samego zestawu wyników o~określonej dokładności.

Metody nieiteracyjne zwykle pozwalają uzyskać wyniki o~dużej dokładności
i~powtarzalności. Ich zastosowanie jest jednak ograniczone do~układów
równań spełniających określone warunki. W~przypadku większych układów
równań ich wydajność może być~niezadowalająca.

Metody iteracyjne operują na~przybliżeniach. Początkowy zestaw danych
uzupełniony musi zostać o~dowolne przybliżenie wartości brakujących
zmiennych. Każda kolejna iteracja (wykonanie algorytmu) wykorzystuje
wyniki poprzedniej i~pozwala na~uzyskanie kolejnego przybliżenia
wyników.

Jeśli proces iteracyjny jest~zbieżny, kolejne iteracje prowadzą
do~uzyskiwania coraz dokładniejszych wyników. Proces taki zwykle
powtarza~się do~momentu, w~którym różnica pomiędzy kolejnymi
przybliżeniami jest~mniejsza niż~zadana wartość graniczna --- wówczas
zakłada~się, że~otrzymano wynik o~określonej dokładności.

Jeśli proces iteracyjny jest~rozbieżny, uzyskanie wyniku metodą
iteracyjną jest~niemożliwe i~może skutkować zapętleniem programu.
Dlatego~też program winien być~zdolny wykryć rozbieżność procesu
iteracyjnego bądź~posiadać ograniczenie dopuszczalnej liczby iteracji,
po~której przekroczeniu proces iteracyjny zostanie~przerwany.


\subsubsection{Metoda prosta jednorównaniowa}

Przykładem metody nieiteracyjnej jest~metoda prosta jednorównaniowa.
Metoda ta analizuje każde z~równań w~układzie indywidualnie. W~związku
z~tym, wykorzystana może~być jedynie w~sytuacji, kiedy jednoznacznie
określone są wartości minimalnej liczby zmiennych niezbędnych
dla~rozwiązania równania.

W~zależności od~rodzaju równania, są to:

\begin{itemize}

	\item dla~równań stanu czynnika --- wartości przynajmniej jednej
		z~par zmiennych $(p, T)$, $(p, h)$, $(p, s)$, $(p, x)$,
		$(T, x)$, $(h, s)$;

	\item dla~równań liniowo-iloczynowych --- wartości wszystkich
		z~wyjątkiem jednej (dowolnej) zmiennej.

\end{itemize}

Wartość zmiennej wynikowej wyznaczana jest~wprost na~podstawie znanych
wartości zmiennych bądź~przy~pomocy elementarnych przekształceń
arytmetycznych. Jeśli znana są wartości większej niż~minimalna liczby
zmiennych, sprawdzana jest słuszność równania (zgodność wyników).

Podstawową zaletą tej~metody jest możliwość jej zastosowania w~stosunku
do~równań stanu czynnika. Nie~znajdzie ona natomiast zastosowania
w~przypadku, gdy~określenie wartości zmiennej rozwiązania wzajemnie
zależnych równań.


\subsubsection{Metoda rozkładu LU}

Metoda rozkładu~LU znajduje zastosowanie jedynie dla~równań liniowych.
Dla~jej zastosowania wyznaczany jest~podzbiór równań, które sprowadzone
mogą zostać do~równań liniowych.

Wyznaczony zostaje wektor rozwiązań wybranego podzbioru równań.
Analizowana jest~również cała przestrzeń rozwiązań w~celu wyróżnienia
układu oznaczonego i~odrzucenia rozwiązań dla~zmiennych należących
do~równań spoza tego podzbioru.

Wykonywane jest również kontrolne mnożenie macierzy współczynników
przez~wektor wyników dla~sprawdzenia poprawności rozwiązań. W~przypadku
zbyt dużej rozbieżności wyniku w~stosunku do~wektora wyrazów wolnych,
syngalizowana jest sprzeczność układu równań.

Zaletą metody rozkładu~LU jest możliwość rozwiązywania szerokiego zbioru
układów liniowych. Podstawową wadą --- brak możliwości wykorzystania
w~stosunku do~równań innego typu.
