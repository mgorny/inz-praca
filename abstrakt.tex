\begin{abstract}

Autor dostrzega potencjał wykorzystania komputerów w~modelowaniu obiegów
cieplnych, którego wykorzystanie nie~jest możliwe w~pełni przy~pomocy
istniejących programów. Proponuje zaprojektowanie i~wykonanie nowego programu,
charakteryzującego~się większą elastycznością. Przedstawia szereg problemów
związanych z~modelowaniem obiegów cieplnych i~proponuje dla~nich rozwiązania.
Dokumentuje algorytmy i~modele wykorzystane w~programie. Bilansuje przykładowy
obieg parowy przy~jego pomocy, a~następnie porównuje wyniki z~rezultatem pracy
programu profesjonalnego oraz~własnych obliczeń. Potwierdza poprawność
konstrukcji programu i~możliwość wykorzystania go do~obliczeń. Przewiduje
dalszą pracę nad~programem, uzupełnienie go o~możliwość modelowania
organicznego obiegu Rankine'a i~obiegów gazowych.

\end{abstract}

\begin{english}
\begin{abstract}

The~author notices the~potential of~using computers in~heat cycle
modelling which can not be~used in~full with the~existing programs.
He proposes to~design and~write a~new program being characterized
by~a~greater deal of~flexibility. He presents a~number of~problems
related to~heat cycle modelling and~proposes their solutions. He
documents the~algorithms and~models used in~his program. He uses it
to~model an~example steam cycle and~compares the~resultant data against
data obtained using professional software and~his own calculations. He
validates the~correctness of~the~program design and~the~possibility
of~using it in~computations. He expects to~continue working
on~the~program, supplementing it with~support for~Organic Rankine Cycle
and~gas turbines.

\end{abstract}
\end{english}
