\section{Program modelowania obiegów}

\subsection{Wybór języka programowania}

Język programowania wykorzystany do~wykonania programu powinien
charakteryzować~się szeregiem cech, ułatwiających potencjalnemu
użytkownikowi instalację i~późniejsze utrzymanie zainstalowanego
programu. Najogólniej określić je można następującymi pojęciami:

\begin{enumerate}

	\item minimalna liczba zewnętrznych zależności (modularność),

	\item rozpowszechnienie programów pisanych w~danym języku,

	\item stabilność języka oraz~infrastruktury koniecznej
	do~uruchamiania programów,

	\item przenośność języka.

\end{enumerate}

Najogólniej założyć można, że~wybór danego języka implikuje szereg
zależności, jakie muszą zostać spełnione nim możliwe będzie uruchomienie
danego programu. Są to~najczęściej tzw.~biblioteki uruchomieniowe,
rzadziej kompletne programy. Modularność języka umożliwia wykorzystanie
minimalnego podzbioru struktury języka, umożliwiającej realizację zadań
bez~konieczności wprowadzenia zależności niezwiązanych z~funkcjami
programu.

Rozpowszechnienie programów wyraża prawdopodobieństwo, że~potencjalny
użytkownik będzie posiadać zainstalowane niezbędne zależności, tym~samym
mogąc uruchomić wykonany program bez~dodatkowych czynności.

Stabilność określa przede~wszystkim zmienność struktury języka w~czasie,
oraz~częstotliwość odnajdywania błędów w~oprogramowaniu towarzyszącym.
Język mało stabilny może wymagać od~programisty regularnego wprowadzania
zmian w~programie, zmuszając użytkownika do~niepotrzebnie częstych
aktualizacji.

Przenośność najogólniej oznacza zdolność programów do~pracy na~znacznej
liczbie platform sprzętowych oraz~systemów operacyjnych. Program
powinien być~zgodny z~systemami operacyjnymi zgodnymi ze~standardem
POSIX oraz~systemem Windows.

Zwrócono uwagę również na~szereg innych cech eksploatacyjnych:

\begin{enumerate}

	\item dostępność niezbędnych bibliotek,

	\item możliwość uzyskania dobrej wydajności oraz~małego zużycia
	pamięci,

	\item możliwość wykonywania rozszerzalnych komponentów.

\end{enumerate}

Ze~względu na~realizowane przez~program zadania, konieczne staje~się
wykorzystanie procedur przeliczeniowych dla~parametrów czynnika
oraz~funkcji matematycznych dla~rozwiązywania układów równań. Dostępność
bibliotek realizujących te~funkcje zmniejsza ilość pracy oraz~redukuje
powielanie kodu. Często wiąże~się również z~wykonaniem danych funkcji
przez~osoby bardziej kompetentne w~danej dziedzinie niż~autor.

Dobra wydajność jest najczęściej cechą charakterystyczną języków
kompilowanych. Ogranicza wymagania sprzętowe programu oraz~czas
oczekiwania na~wyniki.

Rozszerzalność programu oznacza możliwość łatwego dodawania nowych
funkcji i~rozszerzania istniejących komponentów. W~przypadku
realizowanego programu oznacza to możliwość wprowadzenia nowego czynnika
roboczego czy~elementów składowych obiegu. Rozszerzalność realizowana
jest najczęściej przy~pomocy technik programowania obiektowego.

Ze~względu na~wszystkie wyżej wymienione kryteria, autor podjął decyzję
o~wykonaniu programu w~języku C++, w~wariancie opisanym normą
ISO/IEC~14882:1998. Język ten spełnia wszystkie z~postawionych
postulatów.

Do~budowy programu wykorzystane zostały następujące biblioteki
pomocnicze:

\begin{enumerate}

	\item biblioteka funkcji dla~algebry liniowej \textbf{Eigen}, udostępniona
		na~otwartej licencji MPL2,

	\item biblioteka funkcji stanu dla~pary wodnej \textbf{libh2oxx},
		wykonana wcześniej przez~autora na~otwartej licencji BSD.

\end{enumerate}


\subsection{Realizacja elementów obiegu}

Kolejnym istotnym krokiem w~realizacji programu jest wybór sposobu
wykonania przedstawionych wcześniej modeli elementów obiegu. Jest to
zagadnienie ściśle związane z~programowaniem, a~dostępne rozwiązania
uzależnione są w~dużym stopniu od~wyboru języka programowania.

Jednym z~najstarszych i~najbardziej uniwersalnych rozwiązań jest
wykonanie modelu tzw.~,,maszyny uniwersalnej''. Wówczas obieg modelowany
jest za~pomocą szeregu identycznych elementów, które~poprzez wybór
właściwych opcji modelują określone maszyny.

Choć jest to~rozwiązanie względnie proste, jego główną wadą jest
ograniczona uniwersalność. Wprowadzenie każdej kolejnej maszyny wymaga
ingerencji w~model podstawowy, i~czyni go bardziej skomplikowanym.

Rozwiązaniem alternatywnym, charakterystycznym dla~obiektowych języków
programowania, jest wykorzystanie abstrakcyjnych klas bazowych.
Wykonywany wówczas jest podstawowy, abstrakcyjny model ,,maszyny'',
określający jej ogólne cechy charakterystyczne oraz~metody, jakie
model szczegółowy powinien udostępniać.

Poszczególne maszyny wykonywane są w~postaci modeli szczegółowych,
spełniających kryteria postawione przez~model postawowy. Każdy z~takich
modeli zachowuje odrębność kodu, jednocześnie mogąc~być zastosowanym
w~miejsce modelu podstawowego.


\subsection{Metody rozwiązywania układu równań}

\label{rozw-ukl-rown}
