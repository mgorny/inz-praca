\section{Program modelowania obiegów}

\subsection{Wybór języka programowania}

Język programowania wykorzystany do~wykonania programu powinien
charakteryzować~się szeregiem cech, ułatwiających potencjalnemu
użytkownikowi instalację i~późniejsze utrzymanie zainstalowanego
programu. Najogólniej określić je można następującymi pojęciami:

\begin{enumerate}

	\item minimalna liczba zewnętrznych zależności (modularność),

	\item rozpowszechnienie programów pisanych w~danym języku,

	\item stabilność języka oraz~infrastruktury koniecznej
	do~uruchamiania programów,

	\item przenośność języka.

\end{enumerate}

Najogólniej założyć można, że~wybór danego języka implikuje szereg
zależności, jakie muszą zostać spełnione nim możliwe będzie uruchomienie
danego programu. Są to~najczęściej tzw.~biblioteki uruchomieniowe,
rzadziej kompletne programy. Modularność języka umożliwia wykorzystanie
minimalnego podzbioru struktury języka, umożliwiającej realizację zadań
bez~konieczności wprowadzenia zależności niezwiązanych z~funkcjami
programu.

Rozpowszechnienie programów wyraża prawdopodobieństwo, że~potencjalny
użytkownik będzie posiadać zainstalowane niezbędne zależności, tym~samym
mogąc uruchomić wykonany program bez~dodatkowych czynności.

Stabilność określa przede~wszystkim zmienność struktury języka w~czasie,
oraz~częstotliwość odnajdywania błędów w~oprogramowaniu towarzyszącym.
Język mało stabilny może wymagać od~programisty regularnego wprowadzania
zmian w~programie, zmuszając użytkownika do~niepotrzebnie częstych
aktualizacji.

Przenośność najogólniej oznacza zdolność programów do~pracy na~znacznej
liczbie platform sprzętowych oraz~systemów operacyjnych. Program
powinien być~zgodny z~systemami operacyjnymi zgodnymi ze~standardem
POSIX oraz~systemem Windows.

Zwrócono uwagę również na~szereg innych cech eksploatacyjnych:

\begin{enumerate}

	\item dostępność niezbędnych bibliotek,

	\item możliwość uzyskania dobrej wydajności oraz~małego zużycia
	pamięci,

	\item możliwość wykonywania rozszerzalnych komponentów.

\end{enumerate}

Ze~względu na~realizowane przez~program zadania, konieczne staje~się
wykorzystanie procedur przeliczeniowych dla~parametrów czynnika
oraz~funkcji matematycznych dla~rozwiązywania układów równań. Dostępność
bibliotek realizujących te~funkcje zmniejsza ilość pracy oraz~redukuje
powielanie kodu. Często wiąże~się również z~wykonaniem danych funkcji
przez~osoby bardziej kompetentne w~danej dziedzinie niż~autor.

Dobra wydajność jest najczęściej cechą charakterystyczną języków
kompilowanych. Ogranicza wymagania sprzętowe programu oraz~czas
oczekiwania na~wyniki.

Rozszerzalność programu oznacza możliwość łatwego dodawania nowych
funkcji i~rozszerzania istniejących komponentów. W~przypadku
realizowanego programu oznacza to możliwość wprowadzenia nowego czynnika
roboczego czy~elementów składowych obiegu. Rozszerzalność realizowana
jest najczęściej przy~pomocy technik programowania obiektowego.

Ze~względu na~wszystkie wyżej wymienione kryteria, autor podjął decyzję
o~wykonaniu programu w~języku C++, w~wariancie opisanym normą
ISO/IEC~14882:1998. Język ten spełnia wszystkie z~postawionych
postulatów.

Do~budowy programu wykorzystane zostały następujące biblioteki
pomocnicze:

\begin{enumerate}

	\item biblioteka funkcji dla~algebry liniowej \textbf{Eigen}, udostępniona
		na~otwartej licencji MPL2,

	\item biblioteka funkcji stanu dla~pary wodnej \textbf{libh2oxx},
		wykonana wcześniej przez~autora na~otwartej licencji BSD.

\end{enumerate}


\subsection{Realizacja elementów obiegu}

\subsubsection{Wybór metodyki}

Kolejnym istotnym krokiem w~realizacji programu jest wybór sposobu
wykonania przedstawionych wcześniej modeli elementów obiegu. Jest to
zagadnienie ściśle związane z~programowaniem, a~dostępne rozwiązania
uzależnione są w~dużym stopniu od~wyboru języka programowania.

Jednym z~najstarszych i~najbardziej uniwersalnych rozwiązań jest
wykonanie modelu tzw.~,,maszyny uniwersalnej''. Wówczas obieg modelowany
jest za~pomocą szeregu identycznych elementów, które~poprzez wybór
właściwych opcji modelują określone maszyny.

Choć jest to~rozwiązanie względnie proste, jego główną wadą jest
ograniczona uniwersalność. Wprowadzenie każdej kolejnej maszyny wymaga
ingerencji w~model podstawowy, i~czyni go bardziej skomplikowanym.

Rozwiązaniem alternatywnym, charakterystycznym dla~obiektowych języków
programowania, jest wykorzystanie abstrakcyjnych klas bazowych.
W~programie wykonany zostaje wówczas podstawowy, abstrakcyjny obrys
klasy takiej jak~,,maszyna''. Określony zostaje szereg metod, które
powinny~być udostępnione przez~każdą maszynę.

Poszczególne maszyny wówczas wykonywane są w~postaci klas podrzędnych,
które implementują wymagane przez~ogólny model metody. W~ten sposób
stają~się zgodne z~owym modelem i~mogą~być zamiennie z~nim
wykorzystywane. Umożliwia to~stworzenie algorytmu, który w~oparciu
o~ogólny zarys maszyny będzie mógł dokonywać obliczeń dla~każdej
z~maszyn na~jego podstawie stworzonych.

Istotną zaletą takiego rozwiązania jest~uniezależnienie zasadniczego
kodu programu od~implementacji maszyn. Możliwe staje~się wprowadzenie
modeli nowych maszyn bez~konieczności ingerencji w~kod zasadniczej
części programu bądź~innych maszyn. Za~sprawą dziediczenia jest możliwe
również rozszerzanie modeli istniejących maszyn.

Zalety te~sprawiły, iż~zdecydowano~się wykonać program w~technice
obiektowej.


\subsubsection{Modelowanie podstawowych elementów obiegu}

Dla~zamodelowania obiegu termodynamicznego konieczne w~wybranej technice
konieczne jest wykonanie przynajmniej dwóch klas podstawowych ---
reprezentujących odpowiednio wierzchołki i~krawędzie grafu,
bądź~urządzenia i~połączenia pomiędzy nimi.

Klasa podstawowa \textit{Urządzenie} (\textit{Device}) może
reprezentować dowolną maszynę pracującą w~obiegu. W~tym celu umożliwia
dostarczenie zbioru zmiennych charakterystycznych dla~maszynyoraz układu
równań, wiążącego je ze~zmiennymi stanu czynnika i~energii wpływającej
i~wypływającej z~maszyny.

Klasa podstawowa \textit{Połączenie} (\textit{Connection}) stanowi
skierowane powiązanie pomiędzy dwoma maszynami. Operuje na~zmiennych
stanu czynnika bądź~energii przepływającej przez~nie; może dostarczać
również równania wiążące te~zmienne.

Model oparty na~dwóch wyżej wymienionych klasach nie~jest wystarczający
dla~precyzyjnego przedstawienia maszyn posiadających większą liczbę
różnych funkcyjnie wyprowadzeń. Przykład stanowią wymienniki przeponowe,
gdzie konieczne jest~przypisanie połączeń do~obiegu pierwotnego
bądź~wtórnego.

Dlatego też wprowadzono trzecią, pośrednią klasę podstawową,
\textit{Wyprowadzenie} (\textit{Pin}). Każda maszyna posiada szereg
ściśle zdefiniowanych wyprowadzeń, między którymi wykonywane są
połączenia.

\begin{figure}[h]
	\centering
	\begin{tikzpicture}

		{
			\node[heatnode] at (0,0) (K-in) {};
			\node[right] at (K-in.center) {in};
			\node[heatnode] at (3,0) (K-out) {};
			\node[left] at (K-out.center) {out};
			\node[heatnode] at (0,-1) (K-fin) {};
			\node[right] at (K-fin.center) {fuel-in};

			\draw (-.25,-2.5) rectangle (3.25,1.5);
			\node[below] at (1.5,1.5) {\textit{Boiler}};
			\node[above] at (1.5,-2.5) {$\eta_B$};
		}

		\begin{scope}[shift={(5,0)}]
			\node[heatnode] at (0,0) (T-in) {};
			\node[right] at (T-in.center) {in};
			\node[heatnode] at (3,0) (T-out) {};
			\node[left] at (T-out.center) {out};
			\node[heatnode] at (3,-1) (T-eout) {};
			\node[left] at (T-eout.center) {energy-out};

			\draw (-.25,-2.5) rectangle (3.25,1.5);
			\node[below] at (1.5,1.5) {\textit{Turbine}};
			\node[above] at (1.5,-2.5) {$\eta_i$, $\eta_m$};
		\end{scope}

		\begin{scope}[shift={(10,0)}]
			\node[heatnode] at (0,0) (S-in) {};
			\node[right] at (S-in.center) {in};
			\node[heatnode] at (3,0) (S-out) {};
			\node[left] at (S-out.center) {out};
			\node[heatnode] at (0,-1) (S-sin) {};
			\node[right] at (S-sin.center) {sec-in};
			\node[heatnode] at (3,-1) (S-sout) {};
			\node[left] at (S-sout.center) {sec-out};

			\draw (-.25,-2.5) rectangle (3.25,1.5);
			\node[below] at (1.5,1.5) {\textit{Condenser}};
			\node[above] at (1.5,-2.5) {$\Delta T$};
		\end{scope}

		\begin{scope}[-stealth]
			\begin{scope}[blue]
				\draw (K-out) -- (T-in)
					node[above right,pos=.1] {$\mathbf{S}_1$}
					node[above left,pos=.9] {$\mathbf{S}_1^*$};
				\draw (T-out) -- (S-in)
					node[above right,pos=.1] {$\mathbf{S}_2$}
					node[above left,pos=.9] {$\mathbf{S}_2^*$};
				\draw (S-out) -- ++(1,0)
					node[above right,pos=.2] {$\mathbf{S}_3$};
			\end{scope}

			\draw[red] (T-eout) -| ++(.75,-1)
				node[above right,pos=.125] {$P_m$};

			\draw[brown] (S-sout) -| ++(.75,-1)
				node[above right,pos=.125] {$\mathbf{S}_{II}$};
		\end{scope}

		\begin{scope}[stealth-]
			\draw[blue] (K-in) -- ++(-1,0)
				node[above left,pos=.2] {$\mathbf{S}_0^*$};

			\draw[green!50!black] (K-fin) -| ++(-.75,-1)
				node[above left,pos=.125] {$\dot Q_f^*$};

			\draw[brown] (S-sin) -| ++(-.75,-1)
				node[above left,pos=.125] {$\mathbf{S}_I^*$};
			\end{scope}

	\end{tikzpicture}

	\caption{Przykładowy fragment obiegu cieplnego przedstawiony
		za~pomocą klas}
	\label{klasy-przykl}
\end{figure}

Na~rys.~\ref{klasy-przykl} przedstawiony został przykładowy fragment
obiegu cieplnego wykonany w~oparciu o~model trzech klas. Zaprezentowane
zostały trzy wybrane urządzenia, każde posiadające szereg wyprowadzeń
oraz~parametrów.

Należy zwrócić uwagę, iż~zmienne stanu czynnika (w~tym strumień masy) są
cechą wyprowadzenia. Tym~samym, na~obu końcach połączenia występują inne
zmienne stanu. Są one równe co do~wartości, lecz wielkości strumieni
mają przeciwny znak.

W~praktyce pozwala to na~założenie, że~ujemne wartości strumieni
oznaczają energię wpływającą do~maszyny (zużywaną), zaś~dodatnie ---
wypływającą (wytwarzaną).

W~modelu trójklasowym, równania opisujące bilans maszyny mogą
odnosić~się do~jej parametrów oraz~zmiennych stanu wyprowadzeń
tej~maszyny. Równania opisujące połączenia mogą natomiast odwoływać~się
do~zmiennych stanu należących do~wyprowadzeń stanowiących jego końce.
W~obu przypadkach zespół tych wyprowadzeń nazywa~się
\textit{otoczeniem}.


\subsubsection{Możliwość zastosowania alternatywnego czynnika}

Przedstawiony powyżej model może zostać w~prosty sposób rozszerzony
o~możliwość zastosowania alternatywnego czynnika (innego niż~para wodna)
w~obiegu. Możliwe staje~się wówczas modelowanie chociażby organicznego
obiegu Rankine'a.

Aby~możliwe było stosowanie dowolnego czynnika, elementy układu muszą
spełnić następujące założenia:

\begin{enumerate}

	\item maszyny nie~mogą odnosić~się wprost do~właściwości czynnika,
		mogą natomiast wykorzystywać jego uogólnione parametry,

	\item połączenia muszą dostarczać układ równań uzależniony
		od~zastosowanego czynnika,

	\item dodatkowo, urządzenia powinny określać wewnętrzny przepływ
		czynnika pomiędzy wyprowadzeniami.

\end{enumerate}

Pierwsze z~wymienionych wymagań spełnione jest poprzez wprowadzenie
szeregu zmiennych stanu czynnika (sekcja~\ref{czynnik-woda}), włączając
w~to entropię właściwą ($s$) oraz~suchość ($x$). Właściwe operowanie
tymi wielkościami pozwala na~wymuszanie określonych przemian
termodynamicznych bez~konieczności włączania równań czynnika do~modelu
maszyny.

Każde z~połączeń uzupełnione zostało o~dodatkową właściwość,
umożliwiającą wybór czynnika (podklasy klasy bazowej
\textit{Substance}). Do~modelu urządzeń dodano również metodę
określającą, czy~dwa wybrane wyprowadzenia są ze~sobą wewnętrznie
połączone. W~ten sposób możliwe staje~się wyznaczenie drogi wśród
połączeń i~wymuszenie występowania na~niej jednolitego czynnika.


\subsection{Metody rozwiązywania układu równań}

\label{rozw-ukl-rown}
