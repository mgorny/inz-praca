\section{Wstęp}

W~chwili obecnej nie~istnieje wiele programów, których zadaniem
jest~modelowanie obiegów termodynamicznych. Istniejące z~kolei
częstokroć nie~są w~stanie zaspokoić potrzeb studentów. Obok wysokiej
ceny, która stanowi skuteczną przeszkodę w~pozyskaniu programu,
problem stanowi trudność obsługi.

Niejednokrotnie zamodelowanie względnie prostego obiegu sprawia osobie
niedoświadczonej niemałą trudność. Rozwiązanie tych problemów wymaga
nie~tylko intensywnego korzystania z~pomocy, ale~czasem również
wprowadzania przypadkowych zmian metodą prób i błędów, a~w~ostateczności
korzystania z~pomocy osób bardziej doświadczonych.

Celem tej pracy dyplomowej jest~zaprojektowanie, wykonanie
oraz~przetestowanie zestawu bibliotek, które umożliwiłyby modelowanie
obiegów cieplnych turboparowych oraz~wykonywanie obliczeń w~stanie
ustalonym. Powinny one charakteryzować~się:

\begin{enumerate}

\item względną prostotą obsługi, intuicyjnością;

\item dobrą dokładnością obliczeń oraz wydajnością;

\item możliwością modelowania złożonych obiegów cieplnych
oraz~wykonywania obliczeń dla~różnych kombinacji zadanych
i~poszukiwanych parametrów;

\item łatwością wykonywania obliczeń dla~wielu zadanych parametrów
oraz~zestawiania i~porównywania otrzymywanych wyników;

\item możliwością łatwego rozszerzania programu, włączając w~to
możliwość wprowadzenia nowego czynnika.

\end{enumerate}
