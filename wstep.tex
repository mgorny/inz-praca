\section{Wstęp}

Wykonywanie obliczeń bilansowych obiegów cieplnych jest~czynnością
żmudną i~czasochłonną. W~zależności od~złożoności układu, wykonanie
pojedynczego przeliczenia wymaga wyprowadzenia kilku do~kilkunastu
równań, odczytania dwukrotnie większej liczby wartości entalpii
i~rozwiązania tych równań.

Dokładność obliczeń bilansowych wykonywanych w~ten sposób zależna jest
głównie od~dokładnością, z~jaką odczytane zostaną wartości entalpii
właściwej. Zwykle obliczenia czynione z~pomocą wykresu h-s można
traktować jedynie jako obliczenia pobieżne.

Obliczenia tego typu mogą znaleźć głównie zastosowanie przy~wstępnej
bądź~pobieżnej analizie obiegów cieplnych, a~zwłaszcza analizie
porównawczej różnych wariantów obiegu. Paradoksalnie, w~tym przypadku
szczególnie zauważalna staje~się czasochłonność obliczeń, skutecznie
ograniczając liczbę wariantów możliwych do~zrealizowania.

Rozwiązaniem tego problemu jest~wykorzystanie komputerów do~obliczeń.
Wykonanie pojedynczego przeliczenia obiegu cieplnego, włączając w~to
analizę wprowadzonego schematu cieplnego, wyznaczenie wszystkich
parametrów czynnika i~rozwiązanie układu równań, zajmuje mniej
niż~sekundę.

Możliwe staje~się wówczas wykonywanie wielokrotnych przeliczeń
pojedynczego obiegu cieplnego w~prosty sposób, przykładowo w~celu
wyznaczenia charakterystyk sprawności obiegu w~funkcji temperatury pary
świeżej. W~ciągu jednej minuty możliwe jest~wyznaczenie ponad~200
punktów pomiarowych, co~umożliwia wykonanie wykresów o~dużej dokładności
przy~niewielkim nakładzie czasu.

Nie~istnieje współcześnie wiele programów umożliwiających wykonywanie
bilansów układów cieplnych. Wyróżnić można przede~wszystkim aplikacje
małej skali, wykonywane w~miarę potrzeb dla~modelowania wybranego,
pojedynczego obiegu cieplnego. Istnieje również kilka programów bardziej
ogólnych, umożliwiających modelowanie dowolnych obiegów cieplnych.

Na~uwagę zasługuje tu przede~wszystkim popularny w~środowisku
akademickim program Cycle-Tempo. Jest to profesjonalny, komercyjny
program, umożliwiający modelowanie dowolnych obiegów cieplnych. Jednakże
charakteryzuje~się szeregiem wad; dla~rozważanej przez~autora wersji 5.0
wymienić tu można:

\begin{itemize}

	\item trudność i~czasochłonność obsługi, a~w~szczególności
		nieprecyzyjność komunikatów o~błędach,

	\item brak wbudowanych funkcji umożliwiających wykonywanie
		wielokrotnych obliczeń porównawczych,

	\item zamknięty kod źródłowy oraz~struktura programu,
		uniemożliwiające wykonanie brakującej funkcjonalności.

\end{itemize}

Podstawowym celem tej~pracy jest~zaprojektowanie, wykonanie
i~przetestowanie programu, umożliwiającego wykonywanie obliczeń
dowolnych parowych obiegów cieplnych, który~wolny byłby od~wyżej
wymienionych problemów.

Jakkolwiek skala tego przedsięwzięcia jest nieporównywalnie mniejsza
w~stosunku do~rozwiązań profesjonalnych, wykonana w~ramach pracy
dyplomowej aplikacja winna znaleźć zastosowanie dla~prostych obliczeń
obiegów parowych. Przewiduje~się również możliwość rozszerzania programu
o~nowe funkcje i~komponenty w~przyszłości.

Celem drugorzędnym jest~wykonanie ogólnego zestawienia problemów
związanych z~komputerowym modelowaniem obiegów cieplnych
i~przedstawienie możliwych rozwiązań poszczególnych z~nich.
