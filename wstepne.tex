\section{Założenia}

\subsection{Wymagania ogólne}

Podstawowym zadaniem postawionym w~tej~pracy jest~stworzenie programu
modelującego obieg cieplny elektrowni parowej w~celu wykonania obliczeń
bilansowych. Dla~uzyskania programu o~potencjalnie dużym obszarze
zastosowania, postawiono szereg wymagań dodatkowych.

Wykonany program winien charakteryzować~się:

\begin{enumerate}

	\item możliwością modelowania dowolnych obiegów parowych
	bez~konieczności modyfikowania kodu rdzenia programu,

	\item możliwością wykonywania obliczeń przy~dowolnie wybranym
	zbiorze parametrów zadanych oraz~zmiennych szukanych,

	\item możliwością wykonywania obliczeń przy~niepełnym
	bądź~nadmiarowym zestawie danych,

	\item możliwością wprowadzania modeli nowych maszyn w~sposób
	nie~wymagający zmian w~rdzeniu programu bądź~modelu maszyn
	istniejących,

	\item możliwością wykorzystania czynnika termodynamicznego innego
	niż~para wodna bez~konieczności wprowadzania zmian w~rdzeniu
	programu i~modelach maszyn cieplnych.

\end{enumerate}

Spełnienie tych~wymagań umożliwi wykonanie programu o~elastycznej
strukturze, który w~przyszłości będzie mógł zostać rozszerzony o~funkcje
takie jak~modelowanie układów wykorzystujących organiczny obieg
Rankine'a czy~nawet siłowni opartych o~turbiny gazowe.


\subsection{Wybór języka programowania}

\subsubsection{Wymagania}

Język programowania wykorzystany do~wykonania programu powinien zostać
wybrany ze~względu na~szereg cech projektowych i~eksploatacyjnych. Język
taki powinien umożliwić realizację wcześniej postawionych zadań w~sposób
optymalny, nie~stawiając zbędnych wymagań przed~projektantem,
programistą bądź~użytkownikiem.

Najważniejsze właściwości z~punktu widzenia programisty to:

\begin{enumerate}

	\item dostępność niezbędnych bibliotek (funkcji),

	\item możliwość uzyskania dobrej wydajności oraz~małego zużycia
	pamięci,

	\item możliwość wykonywania rozszerzalnych komponentów.

\end{enumerate}

wybranye~względu na~realizowane przez~program zadania, konieczne staje~się
wykorzystanie procedur przeliczeniowych dla~parametrów czynnika
oraz~funkcji matematycznych dla~rozwiązywania układów równań. Dostępność
bibliotek realizujących te~funkcje zmniejsza ilość pracy oraz~redukuje
powielanie kodu. Często wiąże~się również z~wykonaniem danych funkcji
przez~osoby bardziej kompetentne w~danej dziedzinie niż~autor.

Dobra wydajność jest najczęściej cechą charakterystyczną języków
kompilowanych. Ogranicza wymagania sprzętowe programu oraz~czas
oczekiwania na~wyniki.

Rozszerzalność programu oznacza możliwość łatwego dodawania nowych
funkcji i~rozszerzania istniejących komponentów. W~przypadku
realizowanego programu oznacza to możliwość wprowadzenia nowego czynnika
roboczego czy~elementów składowych obiegu. Rozszerzalność realizowana
jest najczęściej przy~pomocy technik programowania obiektowego.

Obok tych cech, język programowania powinien charakteryzować~się również
łatwością instalacji i~utrzymania programu. Wyliczyć można szereg
następujących wymagań:

\begin{enumerate}

	\item minimalna liczba zewnętrznych zależności (modularność),

	\item rozpowszechnienie programów pisanych w~danym języku,

	\item stabilność języka oraz~infrastruktury koniecznej
	do~uruchamiania programów,

	\item przenośność języka.

\end{enumerate}

Najogólniej założyć można, że~wybór danego języka implikuje szereg
zależności, jakie muszą zostać spełnione nim możliwe będzie uruchomienie
danego programu. Są to~najczęściej tzw.~biblioteki uruchomieniowe,
rzadziej kompletne programy. Modularność języka umożliwia wykorzystanie
minimalnego podzbioru struktury języka, umożliwiającej realizację zadań
bez~konieczności wprowadzenia zależności niezwiązanych z~funkcjami
programu.

Rozpowszechnienie programów wyraża prawdopodobieństwo, że~potencjalny
użytkownik będzie posiadać zainstalowane niezbędne zależności,
a~tym~samym będzie mógł uruchomić wykonany program bez~konieczności
wykonywania dodatkowych czynności.

Stabilność określa przede~wszystkim zmienność struktury języka w~czasie,
oraz~częstotliwość odnajdywania błędów w~oprogramowaniu towarzyszącym.
Język mało stabilny może wymagać od~programisty regularnego wprowadzania
zmian w~programie, co~zmusza użytkownika do~niepotrzebnie częstych
aktualizacji.

Przenośność najogólniej oznacza zdolność programów do~pracy na~znacznej
liczbie platform sprzętowych oraz~systemów operacyjnych. Program
powinien być~zgodny z~systemami operacyjnymi zgodnymi ze~standardem
POSIX oraz~systemem Windows.


\subsubsection{Uzasadnienie wyboru}

Ze~względu na~wszystkie wyżej wymienione kryteria, zdecydowano~się
na~realizację programu w~języku \Cpp, w~wariancie opisanym normą
ISO/IEC~14882:1998. Język ten spełnia wszystkie z~postawionych
postulatów.

Jest to~popularny język programowania o~dużej stabilności. Dostępna jest
znaczna liczba implementacji, a~biblioteki standardowe niezbędne
do~uruchamiania aplikacji instalowane są razem z~większością popularnych
systemów operacyjnych.

\Cpp~jest językiem obiektowym oraz~umożliwia budowanie bibliotek. Obie te
cechy pozwalają na~wykonanie programu modularnego, o~strukturze
umożliwiającej rozszerzanie o~nowe funkcje bez~konieczności wprowadzania
zmian w~programie.

Do~budowy programu wykorzystane zostały następujące biblioteki
pomocnicze:

\begin{enumerate}

	\item biblioteka funkcji dla~algebry liniowej \textbf{Eigen}, udostępniona
		na~otwartej licencji MPL2\cite{eigen},

	\item biblioteka funkcji stanu dla~pary wodnej \textbf{libh2oxx},
		wykonana wcześniej przez~autora i~udostępniona na~otwartej
		licencji BSD\cite{libh2o}.

\end{enumerate}


\subsection{Przebieg procesu bilansowania}

Proces wykonywania obliczeń bilansowych ma na~celu wyznaczenie wartości
poszukiwanych zmiennych charakteryzujących określony obieg cieplny,
przy~założeniu zadanego schematu i~parametrów. Najogólniej proces
ten~podzielić można na~dwa etapy:

\begin{enumerate}

\item wykonanie układu równań, opisującego bilans układu;

\item rozwiązanie wykonanego układu równań.

\end{enumerate}

Pierwszy z~wymienionych kroków obejmuje przede~wszystkim analizę
schematu cieplnego układu. Program uwzględnić musi parametry i~sposób
połączenia zainstalowanych maszyn, jak~również dodatkowe informacje
dostarczone przez~użytkownika.

W~oparciu o~te dane utworzony winien zostać układ równań, umożliwiający
jednoznaczne wyznaczenie pozostałych parametrów obiegu. Struktura
poszczególnych równań może być~silnie uzależniona zarówno od~schematu
obiegu, jak~i~wyboru zadanych parametrów. Zastosowane w~programie
algorytmy winne być w~stanie rozwiązać ten~układ, niezależnie od~wyboru
zmiennych zadanych.

Obydwa te~zagadnienia zostaną szerzej omówione w~dalszej części pracy.
