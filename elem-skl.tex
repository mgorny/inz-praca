\section{Modele elementów składowych}

\subsection{Elementy składowe grafu}

Jak~przedstawiono w~poprzednim rozdziale, podstawowymi elementami grafu
w~zastosowanej metodzie są:

\begin{itemize}

	\item wierzchołki, reprezentujące maszyny robocze;

	\item krawędzie, wyrażające przepływ czynnika pomiędzy maszynami.

\end{itemize}

Każdy z~tych elementów modelowany jest w~postaci abstrakcyjnego obiektu,
charakteryzującego~się co~najmniej:

\begin{enumerate}

	\item zbiorem parametrów (zmiennych),

	\item zbiorem równań bilansowych.

\end{enumerate}

W~rozdziale tym zestawione zostaną elementy składowe obiegów cieplnych,
które zastosowane mogą~być w~programie modelowania obiegów cieplnych.


\subsection{Podstawy fizyczne modelu}

Równiania wykorzystywane do~modelowania elementów można najogólniej
podzielić na:

\begin{enumerate}

	\item równania bilansowe,

	\item równania opisujące przemiany termodynamiczne,

	\item równania pomocnicze.

\end{enumerate}

Równania bilansowe oparte są o~podstawowe prawa fizyki.
Do~najważniejszych wykorzystywanych w~modelu należą:

\begin{enumerate}

	\item zasada zachowania masy,

	\item zasada zachowania energii,

	\item I~zasada termodynamiki.

\end{enumerate}

Równania opisujące przemiany termodynamiczne wyrażają zależności
pomiędzy parametrami czynnika przed~maszyną i~za~nią.

\subsection{Modele stanu czynnika}

\subsubsection{Czynnik termodynamiczny --- woda}

\begin{description}

	\item[Parametry:] \hfill

		\begin{itemize}

			\item strumień masy czynnika $\dot m$,

			\item entalpia jednostkowa czynnika $h$,

			\item entropia jednostkowa czynnika $s$,

			\item ciśnienie czynnika $p$,

			\item temperatura czynnika $T$,

			\item suchość pary $x$.

		\end{itemize}

\end{description}

Obsługiwane parametry umożliwiają bilansowanie obiegu przy~wykorzystaniu
typowych danych wejściowych oraz~wykonywanie wykresów. Zależności
pomiędzy nimi dla~pary wodnej zrealizowane zostały w~oparciu o~tablice
,,IAPWS Industrial Formulation 1997''. Program wykorzystuje
implementację zależności IF97 wykonaną wcześniej przez~autora.

Możliwe jest również wykonanie obliczeń dla~czynnika, dla~którego
nie~zostały zaimplementowane właściwe zależności. Konieczne jest wówczas
wprowadzenie wszystkich niezbędnych do~obliczeń parametrów.


\subsubsection{Model paliwa}


\begin{description}

	\item[Parametry:] \hfill

		\begin{itemize}

			\item strumień energii zawartej w~paliwie $\dot Q_f$,

			\item strumień masy [objętości] paliwa $\dot m_f$,

			\item jednostkowa wartość kaloryczna paliwa $q_{LHV}$.

		\end{itemize}

	\item[Równania bilansowe:] \hfill

		\begin{itemize}

			\item $\dot Q_f = \dot m_f q_{LHV}$.

		\end{itemize}

\end{description}

Model paliwa zaprojektowany został z~myślą o~uniwersalności.
W~obliczeniach bilansowych wykorzystywana jest wyłącznie wartość
strumienia energii; dwa pozostałe parametry służą jedynie zwiększeniu
wygody użytkownika.

Ze~względu na~ogólną postać wyrażonego równania, może zostać ono
wykorzystane zarówno dla~paliwa w~postaci stałej (wówczas energia
odniesiona jest do~masy paliwa), jak~i~płynu (energia odniesiona jest
do~objętości). Co~więcej, w~przypadku innych źródeł ciepła parametry
pomocnicze mogą zostać zaniedbane i~zadana może zostać wprost energia
dostarczona do~kotła.


\subsection{Modele maszyn roboczych}

\newcommand\wykresTs[1]{
	\subfloat[][Wykres T-s przemiany]{
		\begin{tikzpicture}
			\begin{axis}[
				xtick=\empty,
				ytick=\empty,
				axis lines=left,
				ylabel=T,
				xlabel=s,
				height=5cm,
				every axis x label/.style={at={(xticklabel cs:.95,.2cm)}},
				every axis y label/.style={at={(yticklabel cs:.9,.2cm)}}
			]
				\addplot+[no marks, smooth] table[x=s, y=T] {wykresy/_nasyc.txt};
				#1
			\end{axis}
		\end{tikzpicture}
	}
}

\subsubsection{Kocioł}

\begin{figure}[H]
	\centering

	\subfloat[][Symbol]{
		\begin{tikzpicture}
			\node[boiler] (K) at (0,0) {K};

			{ [blue]
				\draw[stealth-] (K.s) -- ++(0,-1.5)
					node[pos=.5, left] {$\dot m_i$}
					node[pos=.5, right] {$h_i$};
				\draw[-stealth] (K.n) -- ++(0,1.5)
					node[pos=.5, left] {$\dot m_o$}
					node[pos=.5, right] {$h_o$};
			}

			{ [red, dashed]
				\draw[stealth-] (K.w) -- ++(-2,0)
					node[pos=.5, above] {$\dot Q_f$};
			}
		\end{tikzpicture}
	}%
	~ %
	\wykresTs{
		\addplot+[no marks] table[x=s,y=T] {wykresy/kociol.txt};
	}

	\caption{Model kotła}
	\label{elem:kociol}
\end{figure}

\begin{description}

	\item[Parametry:] \hfill

		\begin{itemize}

			\item sprawność kotła $\eta_B$.

		\end{itemize}

	\item[Przemiany termodynamiczne:] \hfill

		\begin{itemize}

			\item $\mathbf{S_i} \rightarrow \mathbf{S_o}$,
				izobaryczne ogrzewanie, odparowanie i~przegrzanie
				czynnika.

		\end{itemize}

	\item[Równania bilansowe:] \hfill

		\begin{itemize}

			\item \eq$p_i = p_o$,

			\item \eq$\dot m_i = \dot m_o$,

			\item \eq$\dot m_i h_i + \eta_B \dot Q_f = \dot m_o h_o$.

		\end{itemize}

\end{description}

Kocioł zamodelowany jest w~postaci prostej maszyny przepływowej,
w~której energia zawarta w~paliwie wykorzystywana jest do~zwiększenia
entropii czynnika. Model umożliwia uwzględnienie strat energii w~postaci
stałej sprawności.

\subsubsection{Turbina (jednostkowa)}

\begin{figure}[H]
	\centering

	\subfloat[][Symbol]{
		\begin{tikzpicture}
			\node[turbine] (T) at (0,0) {T$_i$};

			{ [blue]
				\draw[stealth-] (T.nw) -- ++(0,1.5)
					node[pos=.5, left] {$\dot m_i$}
					node[pos=.5, right] {$h_i$};
				\draw[-stealth] (T.se) -- ++(0,-1.5)
					node[pos=.5, left] {$\dot m_o$}
					node[pos=.5, right] {$h_o$};
			}

			{ [blue, dashed, looseness=1, bend right=45, scale=.48, rotate=-45]
				\draw[-stealth] (T.ne) to ++(1,1) to ++(-1,1)
					to ++(-1,-1) to ++(1,-1);
					\node[above right] at (T.ne) {$h_o'$};
			}

			{ [red, dashed, double distance=.5mm]
				\draw[-stealth, double] (T.e) -- ++(2,0)
					node[pos=.5, above] {$P_m$};
			}
		\end{tikzpicture}
	}%
	~ %
	\wykresTs{
		\addplot+[no marks, smooth] table[x=s,y=T] {wykresy/turbina.txt}
			node[heatnode, pos=0] (st) {}
			node[pos=.55] (as) {}
			node[pos=.5505] (ae) {}
			node[heatnode, pos=1] (en) {};
		\addplot+[no marks] table[x=s,y=T] {wykresy/turbina-i.txt}
			node[pos=.55] (as-i) {}
			node[pos=.5505] (ae-i) {}
			node[heatnode, pos=1] (en-i) {};

		\node[right] at (st) {$\mathbf{S}_i$};
		\node[right] at (en) {$\mathbf{S}_o$};
		\node[left] at (en-i) {$\mathbf{S}_o'$};

		\draw[brown, thick, -latex] (as-i.center) -- (ae-i.center);
		\draw[red, thick, -latex] (as.center) -- (ae.center);
	}

	\caption{Model turbiny}
	\label{elem:turbina}
\end{figure}

\begin{description}

	\item[Parametry:] \hfill

		\begin{itemize}

			\item sprawność izentropowa turbiny $\eta_i$,

			\item sprawność mechaniczna turbiny $\eta_m$.

		\end{itemize}

	\item[Przemiany termodynamiczne:] \hfill

		\begin{itemize}

			\item $\mathbf{S_i} \rightarrow \mathbf{S_o}$,
				adiabatyczne rozprężanie czynnika.

		\end{itemize}

	\item[Równania bilansowe:] \hfill

		\begin{itemize}

			\item \eq$\dot m_i = \dot m_o$,

			\item \eq$p_o' = p_o$,%
				\label{turb.petla.p}

			\item \eq$s_o' = s_i$,%
				\label{turb.petla.s}

			\item \eq$h_o = h_i + \eta_i ( h_o' - h_i )$,%
				\label{turb.wyj.h}

			\item \eq$\dot m_i h_i = \dot m_o h_o + \frac{P_m}{\eta_m}$.

		\end{itemize}

\end{description}

Model turbiny realizuje proces rozprężenia czynnika od~stanu
początkowego do~zadanego ciśnienia. Proces rozprężania dzieli~się
umyślnie na~dwa etapy:

\begin{enumerate}

	\item izentropowe rozprężanie czynnika do~zadanego ciśnienia $p_o$,

	\item wprowadzenie korekty ze~względu na~nieizentropowość
		rzeczywistego procesu rozprężania.

\end{enumerate}

Realizacji pierwszego z~wymienionych etapów dokonuje~się z~pomocą
tzw.~pętli izentropowej. Równania \eqref{turb.petla.p}
i~\eqref{turb.petla.s} wymuszają zamodelowanie stanu czynnika
po~izentropowym rozprężaniu $\mathbf{S}_o'(p_o, s_o'=s_i)$.

Następnie, równanie~\eqref{turb.wyj.h} wykorzystywane jest
do~wyznaczenia entalpii czynnika po~rzeczywistym rozprężaniu $h_o$,
przy~uwzględnieniu zadanej sprawności izentropowej $\eta_i$. Wyznaczony
może zostać wyjściowy stan czynnika $\mathbf{S}_o(p_o, h_o)$.

Model turbiny jednostkowej charakteryzuje~się pojedynczym wyjściem
czynnika. Realizacja turbiny upustowo-kondensacyjnej
oraz~upustowo-przeciwprężnej wymaga równoległego połączenia elementów,
z~uwzględnieniem rozgałęzienia czynnika, oraz~węzła sumacyjnego
dla~wyprowadzeń mocy. Schematycznie układ taki zaprezentowany został
na~rys.~\ref{turb.upustowa}.

\begin{figure}[H]
	\centering

	\begin{tikzpicture}
		\node[turbine] (T1) at (0,0) {T$_1$};
		\node[turbine] (T2) at (4,0) {T$_2$};
		\node[turbine] (T3) at (8,0) {T$_3$};

		{ [blue]
			\draw[stealth-] (T1.nw) -- ++(0,1.5)
				node[pos=.75, left] {$\dot m_i$}
				node[pos=.75, right] {$h_i$};

			\path (T1.nw) -- ++(0,.75) node[heatnode] (T1x) {};
			\path (T2.nw) -- ++(0,.75) node[heatnode] (T2x) {};

			\draw[-stealth] (T1x) -| (T2.nw);
			\draw[-stealth] (T2x) -| (T3.nw);

			\draw[-stealth] (T1.se) -- ++(0,-1.5)
				node[pos=.5, left] {$\dot m_{b1}$}
				node[pos=.5, right] {$h_{b1}$};

			\draw[-stealth] (T2.se) -- ++(0,-1.5)
				node[pos=.5, left] {$\dot m_{b2}$}
				node[pos=.5, right] {$h_{b2}$};

			\draw[-stealth] (T3.se) -- ++(0,-1.5)
				node[pos=.5, left] {$\dot m_o$}
				node[pos=.5, right] {$h_o'$};
		}

		{ [red, double distance=.5mm]
			\node[sumnode] (S2) at (6.5,1.75) {};
			\node[sumnode] (S3) at (10.5,1.75) {};

			\draw[-stealth, double] (T1.e) -| (2.5,1.75)
				node[pos=0, above right] {$P_{m1}$}
				-- (S2.w);

			\draw[-stealth, double] (T2.e) -| (S2.s)
				node[pos=0, above right] {$P_{m2}$};

			\draw[-stealth, double] (S2.e) -- (S3.w);

			\draw[-stealth, double] (T3.e) -| (S3.s)
				node[pos=0, above right] {$P_{m3}$};

			\draw[-stealth, double] (S3.e) -- ++(1.5,0)
				node[pos=1, above left] {$P_m$};
		}
	\end{tikzpicture}

	\caption{Schematyczne przedstawienie połączeń dla~modelowania
		turbiny upustowo-kondensacyjnej}
	\label{turb.upustowa}
\end{figure}

\subsubsection{Skraplacz}

\begin{figure}[H]
	\centering

	\subfloat[][Symbol]{
		\begin{tikzpicture}
			\node[condenser] (Sk) at (0,0) {Sk};

			{ [blue]
				\draw[stealth-] (Sk.n) -- ++(0,1.5)
					node[pos=.5, left] {$\dot m_i$}
					node[pos=.5, right] {$h_i$}
					node[pos=1] (tn) {};
				\draw[-stealth] (Sk.s) -- ++(0,-1.5)
					node[pos=.5, left] {$\dot m_o$}
					node[pos=.5, right] {$h_o$}
					node[pos=1] (ts) {};
			}

			{ [green!50!black]
				\draw[-stealth] (tn) ++(1.75,0) |- (Sk.ne)
					node[pos=.225, left] {$\dot m_i'$}
					node[pos=.225, right] {$h_i'$};
				\draw[stealth-] (ts) ++(1.75,0) |- (Sk.se)
					node[pos=.225, left] {$\dot m_o'$}
					node[pos=.225, right] {$h_o'$};
			}
		\end{tikzpicture}
	}%
	~ %
	\wykresTs{
		\addplot+[no marks] table[x=s,y=T] {wykresy/skraplacz.txt}
			node[heatnode, pos=0] (st) {}
			node[pos=.5] (as) {}
			node[pos=.5005] (ae) {}
			node[heatnode, pos=1] (en) {};

		\node[right] at (st) {$\mathbf{S}_i$};
		\node[right] at (en) {$\mathbf{S}_o$};

		\draw[red, thick, -latex] (as.center) -- (ae.center);
	}

	\caption{Model skraplacza}
	\label{elem:skraplacz}
\end{figure}

\begin{description}

	\item[Parametry:] \hfill

		\begin{itemize}

			\item różnica temperatur na~wyprowadzeniach obiegu wtórnego
				$\Delta T$.

		\end{itemize}

	\item[Przemiany termodynamiczne:] \hfill

		\begin{itemize}

			\item $\mathbf{S_i} \rightarrow \mathbf{S_o}$,
				izobaryczne skraplanie czynnika;

			\item $\mathbf{S_i'} \rightarrow \mathbf{S_o'}$,
				izobaryczne ogrzewanie czynnika chłodzącego.

		\end{itemize}

	\item[Równania bilansowe:] \hfill

		\begin{itemize}

			\item \eq$\dot m_i = \dot m_o$,

			\item \eq$\dot m_i' = \dot m_o'$,

			\item \eq$p_i = p_o$,

			\item \eq$p_i' = p_o'$,

			\item \eq$x_o = 0$,%
				\label{skraplacz-x0}

			\item \eq$T_o' = T_i' + \Delta T$,%
				\label{skraplacz-DT}

			\item \eq$\dot m_i h_i + \dot m_i' h_i'
				= \dot m_o h_o + \dot m_o' h_o'$.

		\end{itemize}

\end{description}

Skraplacz modelowany jest jako~przeponowy wymiennik ciepła pracujący
w~dwóch obiegach. Model uwzględnia izobaryczną wymianę ciepła pomiędzy
czynnikami bez~ich mieszania. Równanie~\eqref{skraplacz-x0} dodatkowo
wymusza skroplenie czynnika roboczego (w~obiegu pierwotnym).

Dla~uproszczenia typowych obliczeń wprowadzona została dodatkowa
wielkość w~postaci różnicy temperatur czynnika na~początku i~na~końcu
obiegu chłodzącego. Przyjmuje ona domyślnie wartość \SI{10}{\kelvin}.

