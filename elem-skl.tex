\section{MODELE ELEMENTÓW SKŁADOWYCH}

\subsection{Elementy składowe grafu}

Jak~przedstawiono w~poprzednim rozdziale, podstawowymi elementami grafu
w~zastosowanej metodzie są:

\begin{itemize}

	\item wierzchołki, reprezentujące maszyny robocze;

	\item krawędzie, wyrażające przepływ czynnika pomiędzy maszynami.

\end{itemize}

Każdy z~tych elementów modelowany jest w~postaci abstrakcyjnego obiektu,
charakteryzującego~się co~najmniej:

\begin{enumerate}

	\item zbiorem parametrów (zmiennych),

	\item zbiorem równań bilansowych.

\end{enumerate}

W~rozdziale tym zestawione zostaną elementy składowe obiegów cieplnych,
które zastosowane mogą~być w~programie modelowania obiegów cieplnych.


\subsection{Podstawy fizyczne modelu}

Podstawą fizyczną dla~wszystkich równań bilansowych są~tzw.~,,zasady
zachowania'':

\begin{enumerate}

	\item zasada zachowania masy,

	\item zasada zachowania pędu,

	\item zasada zachowania momentu pędu,

	\item zasada zachowania energii,

	\item zasada bilansu entropii.

\end{enumerate}

TODO: rozwinąć, wzorki


\subsection{Modele stanu czynnika}

\subsubsection{Czynnik termodynamiczny -- woda}

Parametry:

\begin{enumerate}

	\item strumień masy czynnika $\dot m$,

	\item entalpia jednostkowa czynnika $h$,

	\item entropia jednostkowa czynnika $s$,

	\item ciśnienie czynnika $p$,

	\item temperatura czynnika $T$.

\end{enumerate}

Zależności pomiędzy parametrami czynnika zrealizowane zostały w~oparciu
o~tablice ,,IAPWS Industrial Formulation 1997''. Program wykorzystuje
implementację zależności IF97 wykonaną wcześniej przez~autora.


\subsubsection{Model paliwa}

Parametry:

\begin{enumerate}

	\item strumień energii zawartej w~paliwie $\dot Q_f$,

	\item strumień masy paliwa $\dot m_f$,

	\item wartość kaloryczna paliwa $q_{LHV}$.

\end{enumerate}

Równania bilansowe:

\begin{enumerate}

	\item $\dot Q_f = \dot m_f q_{LHV}$.

\end{enumerate}


\subsection{Modele maszyn roboczych}

\subsubsection{Kocioł}

\begin{wrapfigure}{r}{4cm}
	\centering

	\begin{tikzpicture}
		\node[boiler] (K) at (0,0) {K};

		{ [blue]
			\draw[stealth-] (K.s) -- ++(0,-1)
				node[pos=.5, left] {$\dot m_i$}
				node[pos=.5, right] {$h_i$};
			\draw[-stealth] (K.n) -- ++(0,1)
				node[pos=.5, left] {$\dot m_o$}
				node[pos=.5, right] {$h_o$};
		}

		{ [red, dashed]
			\draw[stealth-] (K.w) -- ++(-1.5,0)
				node[pos=.5, above] {$\dot m_f$}
				node[pos=.5, below] {$q_{LHV}$};
		}
	\end{tikzpicture}
\end{wrapfigure}

Parametry:

\begin{enumerate}

	\item sprawność kotła $\eta_B$.

\end{enumerate}

Równania bilansowe:

\begin{enumerate}

	\item $\dot m_i = \dot m_o$,

	\item $\dot m_i h_i = \eta_B \dot m_f q_{LHV} + \dot m_o h_o$.

\end{enumerate}
