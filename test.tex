\section{Testowanie programu}

Do~testowania programu wykorzystano uproszczony model bloku
\SI{90}{\mega\watt}, oparty na~schemacie cieplnym zawartym
w~\cite[s.~121]{podstawy_proj}.

Wykorzystany schemat cieplny układu wraz z~zaznaczonymi zadanymi
parametrami przedstawiony został na~rys.~\ref{test-schemat}.

\begin{figure}[h]
	\centering
	\begin{tikzpicture}[thick]
		\begin{scope}
			\node[boiler] (K1) at (-1,1) {K};
			\node[reheater] (K2) at (.5,1) {};

			\node[turbine] (WP) at (4,1.5) {WP};
			\node[turbine] (NP) at (7,1.5) {NP};

			\begin{scope}[font={\scriptsize}]
				\node[condenser] (Sk) at (7.75,-2.75)
					{$\Delta T = \SI{10}{\kelvin}$};
				\node[heatnode] (Skm) at (7.75,-4) {};
				\node[pumpd] (PSk) at (7.75,-4.5) {};

				\node[heatxchg] (R5) at (6.5,-5) {R5};
				\node[heatxchg] (R4) at (5,-5) {R4};
				\node[deaerator] (R3) at (3.25,-3) {R3};
				\node[heatxchg] (R2) at (1.75,-5) {R2};
				\node[heatxchg] (R1) at (0,-5) {R1};
			\end{scope}

			\node[heatnode] (R3m) at (3.25,-4) {};
			\node[pumpd] (P3) at (3.25,-4.5) {};

			\node[gen] (G) at (9,1.5) {$\sim$};
			\node[above, font={\scriptsize}] at (G.n)
				{\SI{90115}{\kilo\watt}};
		\end{scope}

		\begin{scope}[font={\scriptsize},
			inner sep=.75mm
		]
			\begin{scope}[-stealth]
				\draw (K1.n) -- ++(0,.5) -| (WP.nw)
					node[pos=0, above right] {\SI{10,2}{\mega\pascal}}
					node[pos=0, below right] {\SI{538}{\degreeCelsius}}
					node[pos=.3, draw, fill=white] {1};
				\draw (K2.n) -- ++(0,1) -| (NP.nw)
					node[pos=0, above right] {\SI{538}{\degreeCelsius}}
					node[pos=.25, draw, fill=white] {3};
			\end{scope}

			\draw (WP.se) -- ++(0,-.4) -| (K2.s)
				node[pos=.499, above right] {\SI{2,35}{\mega\pascal}}
				node[pos=.25, draw, fill=white] {2};
			\draw[-stealth] (NP.se) -- (Sk.n)
				node[pos=.95, sloped, above left] {\SI{3,5}{\kilo\pascal}}
				node[pos=.5, draw, fill=white] {4};

			\draw (Sk.s) -- (PSk.n)
				node[pos=.375, draw, fill=white] {5};
			\draw (PSk.s) |- (R5.e)
				node[pos=.825, draw, fill=white] {6};
			\draw (R5.w) -- (R4.e)
				node[pos=.5, draw, fill=white] {7};
			\draw[-stealth] (R4.w) -- ++(-.25,0) |- (R3.ne)
				node[pos=.25, draw, fill=white] {8};
			\draw (R3.s) -- (P3.n)
				node[pos=.375, draw, fill=white] {9};
			\draw (P3.s) |- (R2.e)
				node[pos=.8, draw, fill=white] {10};
			\draw (R2.w) -- (R1.e)
				node[pos=.5, draw, fill=white] {11};
			\draw (R1.w) -| (K1.s)
				node[pos=.825, draw, fill=white] {12};

			% upusty
			\path (NP.se) -- (NP.sw)
				node[pos=.25] (u5) {}
				node[pos=.5] (u4) {}
				node[pos=.75] (u3) {}
				node[pos=1] (u2) {};
			\path (WP.se) -- (WP.sw)
				node[pos=.5] (u1) {};

			\begin{scope}[-stealth]
				\draw (u5.center) -- ++(0,-2) -| (R5.n)
					node[pos=.95, sloped, above left]
					{\SI{0,07}{\mega\pascal}}
					node[pos=.55, draw, fill=white] {u5};
				\draw (u4.center) -- ++(0,-1.8) -| (R4.n)
					node[pos=.95, sloped, above left]
					{\SI{0,2}{\mega\pascal}}
					node[pos=.55, draw, fill=white] {u4};
				\draw (u3.center) -- ++(0,-1.6) -| (R3.n)
					node[pos=.95, sloped, above left]
					{\SI{0,5}{\mega\pascal}}
					node[pos=.45, draw, fill=white] {u3};
				\draw (u2.center) -- ++(0,-1.4) -| (R2.n)
					node[pos=.95, sloped, above left]
					{\SI{1,37}{\mega\pascal}}
					node[pos=.55, draw, fill=white] {u2};
				\draw (u1.center) -- ++(0,-0.9) -| (R1.n)
					node[pos=.95, sloped, above left]
					{\SI{2,56}{\mega\pascal}}
					node[pos=.55, draw, fill=white] {u1};

				\path (Skm) -- ++(.65,0) node (SkmE) {};
				\path (R5.s) -- ++(0,-.5) node[heatnode] (R5s) {};
				\draw (R5.s) -- ++(0,-.5) -| (SkmE.center)
					node[pos=.25, draw, fill=white] {k5}
					-- (Skm);
				\draw (R4.s) -- ++(0,-.5) -- (R5s)
					node[pos=.5, draw, fill=white] {k4};

				\path (R3m) -- ++(.65,0) node (R3mE) {};
				\path (R2.s) -- ++(0,-.5) node[heatnode] (R2s) {};
				\draw (R2.s) -- ++(0,-.5) -| (R3mE.center)
					node[pos=.25, draw, fill=white] {k2}
					-- (R3m);
				\draw (R1.s) -- ++(0,-.5) -- (R2s)
					node[pos=.5, draw, fill=white] {k1};
			\end{scope}

			% wały
			\draw[double] (WP.e) -- (NP.w);
			\draw[double] (NP.e) -- (G.w);
		\end{scope}
	\end{tikzpicture}

	\caption{Schemat cieplny obiegu testowego}
	\label{test-schemat}
\end{figure}

Obok parametrów zestawionych na~schemacie, założono:

\begin{itemize}
	\item różnicę temperatur pomiędzy skroplinami i~czynnikiem
		podgrzewanym przez~podgrzewacz regeneracyjny na~poziomie
		$\Delta T = \SI{5}{\kelvin}$,
	\item sprawność kotła $\eta_B = \num{0,9}$,
	\item izentropową sprawność maszyn wirujących $\eta_i = \num{0,8}$,
	\item mechaniczną sprawność maszyn wirujących $\eta_m = \num{0,99}$,
	\item wartość opałową paliwa $q_{LHV} =
		\SI{22000}{\kilo\joule\per\kilogram}$.
\end{itemize}

Obieg zamodelowany został przy~pomocy programu wykonanego przez~autora
oraz~programu Cycle-Tempo~5.0. Wyniki symulacji i~obliczeń, a~także
dodatkowe parametry podane w~schemacie umieszczonym w~literaturze
zestawione zostały w~tabeli~\ref{test-wyniki1}.

\begin{longtable}{|*{10}{r|}}
	\caption{Zestawienie porównawcze parametrów czynnika}
	\label{test-wyniki1} \\

	\hline
		\multirow{2}{*}{Rurociąg} &
		&
		\multicolumn{2}{c|}{Program} &
		\multicolumn{2}{c|}{Cycle-Tempo} &
		\multicolumn{1}{c|}{Lit.} \\
	\cline{3-7}
		&
		\multicolumn{1}{c|}{$p$ [\si{\mega\pascal}]} &
		\multicolumn{1}{c|}{$T$ [\si{\degreeCelsius}]} &
		\multicolumn{1}{c|}{$h$ [\si{\kilo\joule\per\kilogram}]} &
		\multicolumn{1}{c|}{$T$ [\si{\degreeCelsius}]} &
		\multicolumn{1}{c|}{$h$ [\si{\kilo\joule\per\kilogram}]} &
		\multicolumn{1}{c|}{$T$ [\si{\degreeCelsius}]} \\
	\hline
	\endfirsthead
	\caption{Zestawienie porównawcze parametrów czynnika (c.d.)} \\

	\hline
		\multirow{2}{*}{Rurociąg} &
		&
		\multicolumn{2}{c|}{Program} &
		\multicolumn{2}{c|}{Cycle-Tempo} &
		\multicolumn{1}{c|}{Lit.} \\
	\cline{3-7}
		&
		\multicolumn{1}{c|}{$p$ [\si{\mega\pascal}]} &
		\multicolumn{1}{c|}{$T$ [\si{\degreeCelsius}]} &
		\multicolumn{1}{c|}{$h$ [\si{\kilo\joule\per\kilogram}]} &
		\multicolumn{1}{c|}{$T$ [\si{\degreeCelsius}]} &
		\multicolumn{1}{c|}{$h$ [\si{\kilo\joule\per\kilogram}]} &
		\multicolumn{1}{c|}{$T$ [\si{\degreeCelsius}]} \\
	\hline
	\endhead
	\hline
	\endfoot
		 1 & 10,2000 & 538,00 & 3469,74 & 538,00 & 3469,74 & 538 \\
		 2 &  2,3500 & 345,08 & 3119,02 & 345,08 & 3119,02 & b/d \\
		 3 &  2,3500 & 538,00 & 3548,83 & 538,00 & 3548,83 & 538 \\
		 4 &  0,0035 &  26,67 & 2495,38 &  26,67 & 2495,38 & b/d \\
		 5 &  0,0035 &  26,67 &  111,84 &  26,67 &  111,84 &  33 \\
		 6 &  0,5000 &  26,76 &  112,57 &  26,71 &  112,46 & b/d \\
		 7 &  0,5000 &  84,93 &  356,01 &  84,93 &  356,01 &  84 \\
		 8 &  0,5000 & 115,21 &  483,69 & 115,21 &  483,69 & 116 \\
		 9 &  0,5000 & 151,84 &  640,19 & 151,84 &  840,19 & 146 \\
		10 & 10,2000 & 153,57 &  653,53 & 153,52 &  653,40 & b/d \\
		11 & 10,2000 & 189,04 &  807,53 & 189,04 &  807,52 & 191 \\
		12 & 10,2000 & 220,22 &  946,94 & 220,22 &  946,94 & 231 \\
		u1 &  2,5600 & 354,70 & 3136,47 & 355,25 & 3137,72 & b/d \\
		k1 &  2,5600 & 225,22 &  967,87 & 225,22 &  967,87 & b/d \\
		u2 &  1,3700 & 464,86 & 3398,56 & 480,50 & 3432,51 & b/d \\
		k2 &  1,3700 & 194,04 &  825,61 & 194,04 &  825,61 & b/d \\
		u3 &  0,5000 & 346,95 & 3161,74 & 379,07 & 3228,55 & b/d \\
		u4 &  0,2000 & 258,91 & 2989,18 & 293,45 & 3058,83 & b/d \\
		k4 &  0,2000 & 120,21 &  504,68 & 120,21 &  504,68 & b/d \\
		u5 &  0,0700 & 177,35 & 2832,41 & 203,49 & 2883,72 & b/d \\
		k5 &  0,0700 &  89,93 &  376,68 &  89,93 &  376,68 & b/d \\
\end{longtable}
