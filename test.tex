\section{Testowanie programu}

\subsection{Dane wejściowe}

Do~testowania programu wykorzystano uproszczony model bloku
\SI{90}{\mega\watt}, oparty na~schemacie cieplnym zawartym
w~\cite[s.~121]{podstawy_proj}.

Wykorzystany schemat cieplny układu wraz z~zaznaczonymi zadanymi
parametrami przedstawiony został na~rys.~\ref{test-schemat}.

\begin{figure}[h]
	\centering
	\begin{tikzpicture}[thick]
		\begin{scope}
			\node[boiler] (K1) at (-1.15,1) {K};
			\node[reheater] (K2) at (.5,1) {};

			\node[turbine] (WP) at (4,1.5) {WP};
			\node[turbine] (NP) at (7.5,1.5) {NP};

			\begin{scope}[font={\scriptsize}]
				\node[condenser] (Sk) at (8.25,-2.75)
					{$\Delta T = \SI{10}{\kelvin}$};
				\node[pumpd] (PSk) at (8.25,-4.5) {};

				\node[heatxchg] (R5) at (6.5,-5) {R5};
				\node[heatxchg] (R4) at (5,-5) {R4};
				\node[deaerator] (R3) at (3.25,-3) {R3};
				\node[heatxchg] (R2) at (1.25,-5) {R2};
				\node[heatxchg] (R1) at (-.4,-5) {R1};
			\end{scope}

			\node[pumpd] (P3) at (3.25,-4.5) {};

			\node[gen] (G) at (9.5,1.5) {$\sim$};
			\node[above, font={\scriptsize}] at (G.n)
				{\SI{90115}{\kilo\watt}};
		\end{scope}

		\begin{scope}[font={\scriptsize},
			inner sep=.75mm
		]
			\begin{scope}[-stealth]
				\draw (K1.n) -- ++(0,.5) -| (WP.nw)
					node[pos=0, above right] {\SI{10,2}{\mega\pascal}}
					node[pos=0, below right] {\SI{538}{\degreeCelsius}}
					node[pos=.3, draw, fill=white] {1};
				\draw (K2.n) -- ++(0,1) -| (NP.nw)
					node[pos=0, above right] {\SI{538}{\degreeCelsius}}
					node[pos=.25, draw, fill=white] {3};
			\end{scope}

			\draw (WP.se) -- ++(0,-.4) -| (K2.s)
				node[pos=.499, above right] {\SI{2,35}{\mega\pascal}}
				node[pos=.25, draw, fill=white] {2};
			\draw[-stealth] (NP.se) -- (Sk.n)
				node[pos=.95, sloped, above left] {\SI{3,5}{\kilo\pascal}}
				node[pos=.5, draw, fill=white] {4};

			\draw (Sk.s) -- (PSk.n)
				node[pos=.375, draw, fill=white] {5};
			\draw (PSk.s) |- (R5.e)
				node[pos=.875, draw, fill=white] {6};
			\draw (R5.w) -- (R4.e)
				node[pos=.5, draw, fill=white] {7};
			\draw[-stealth] (R4.w) -- ++(-.25,0) |- (R3.ne)
				node[pos=.25, draw, fill=white] {8};
			\draw (R3.s) -- (P3.n)
				node[pos=.375, draw, fill=white] {9};
			\draw (P3.s) |- (R2.e)
				node[pos=.7, draw, fill=white] {10};
			\draw (R2.w) -- (R1.e)
				node[pos=.5, draw, fill=white] {11};
			\draw (R1.w) -| (K1.s)
				node[pos=.825, draw, fill=white] {12};

			% upusty
			\path (NP.se) -- (NP.sw)
				node[pos=.25] (u5) {}
				node[pos=.5] (u4) {}
				node[pos=.75] (u3) {}
				node[pos=1] (u2) {};
			\path (WP.se) -- (WP.sw)
				node[pos=.5] (u1) {};

			\begin{scope}[-stealth]
				\draw (u5.center) -- ++(0,-2) -| (R5.n)
					node[pos=.95, sloped, above left]
					{\SI{0,07}{\mega\pascal}}
					node[pos=.55, draw, fill=white] {u5};
				\draw (u4.center) -- ++(0,-1.8) -| (R4.n)
					node[pos=.95, sloped, above left]
					{\SI{0,2}{\mega\pascal}}
					node[pos=.55, draw, fill=white] {u4};
				\draw (u3.center) -- ++(0,-1.6) -| (R3.n)
					node[pos=.95, sloped, above left]
					{\SI{0,5}{\mega\pascal}}
					node[pos=.45, draw, fill=white] {u3};
				\draw (u2.center) -- ++(0,-1.4) -| (R2.n)
					node[pos=.95, sloped, above left]
					{\SI{1,37}{\mega\pascal}}
					node[pos=.55, draw, fill=white] {u2};
				\draw (u1.center) -- ++(0,-0.9) -| (R1.n)
					node[pos=.95, sloped, above left]
					{\SI{2,56}{\mega\pascal}}
					node[pos=.55, draw, fill=white] {u1};

				\path (Sk.sw) -- ++(-.25,-.25) node (SkSW) {};
				\draw (R5.se) -- ++(0,-.5) -| (SkSW.center)
					node[pos=.25, draw, fill=white] {k5}
					-- (Sk.sw);
				\draw (R4.se) -- ++(0,-.5) -| (R5.sw)
					node[pos=.25, draw, fill=white] {k4};

				\path (R3.nw) -- ++(-.7,0) node (R3NW) {};
				\draw (R2.se) -- ++(0,-.5) -| (R3NW.center)
					node[pos=.25, draw, fill=white] {k2}
					-- (R3.nw);
				\draw (R1.se) -- ++(0,-.5) -| (R2.sw)
					node[pos=.25, draw, fill=white] {k1};
			\end{scope}

			% wały
			\draw[double] (WP.e) -- (NP.w);
			\draw[double] (NP.e) -- (G.w);
		\end{scope}
	\end{tikzpicture}

	\caption{Schemat cieplny obiegu testowego}
	\label{test-schemat}
\end{figure}

Obok parametrów zestawionych na~schemacie, założono:

\begin{itemize}
	\item różnicę temperatur pomiędzy skroplinami i~czynnikiem
		podgrzewanym przez~podgrzewacz regeneracyjny na~poziomie
		$\Delta T = \SI{5}{\kelvin}$,
	\item sprawność kotła $\eta_B = \num{0,9}$,
	\item izentropową sprawność maszyn wirujących $\eta_i = \num{0,8}$,
	\item mechaniczną sprawność maszyn wirujących $\eta_m = \num{0,99}$.
\end{itemize}


\subsection{Zestawienie wyników}

Obieg zamodelowany został przy~pomocy programu wykonanego przez~autora
oraz~komercyjnego programu Cycle-Tempo~5.0. W~tabeli~\ref{test-wyniki1}
zestawiono parametry czynnika przepływającego przez~poszczególne
rurociągi, wyznaczone przy~pomocy obu programów oraz~dodatkowe
informacje, zawarte na~schemacie cieplnym w~literaturze.

\begin{longtable}{|*{10}{r|}}
	\caption{Zestawienie porównawcze parametrów czynnika}
	\label{test-wyniki1} \\

	\hline
		\multirow{2}{*}{Rurociąg} &
		\multicolumn{1}{c|}{Zadane} &
		\multicolumn{2}{c|}{Program} &
		\multicolumn{2}{c|}{Cycle-Tempo} &
		\multicolumn{1}{c|}{Lit.} \\
	\cline{2-7}
		&
		\multicolumn{1}{c|}{$p$ [\si{\mega\pascal}]} &
		\multicolumn{1}{c|}{$T$ [\si{\degreeCelsius}]} &
		\multicolumn{1}{c|}{$h$ [\si{\kilo\joule\per\kilogram}]} &
		\multicolumn{1}{c|}{$T$ [\si{\degreeCelsius}]} &
		\multicolumn{1}{c|}{$h$ [\si{\kilo\joule\per\kilogram}]} &
		\multicolumn{1}{c|}{$T$ [\si{\degreeCelsius}]} \\
	\hline
	\endfirsthead
	\caption{Zestawienie porównawcze parametrów czynnika (c.d.)} \\

	\hline
		\multirow{2}{*}{Rurociąg} &
		\multicolumn{1}{c|}{Zadane} &
		\multicolumn{2}{c|}{Program} &
		\multicolumn{2}{c|}{Cycle-Tempo} &
		\multicolumn{1}{c|}{Lit.} \\
	\cline{2-7}
		&
		\multicolumn{1}{c|}{$p$ [\si{\mega\pascal}]} &
		\multicolumn{1}{c|}{$T$ [\si{\degreeCelsius}]} &
		\multicolumn{1}{c|}{$h$ [\si{\kilo\joule\per\kilogram}]} &
		\multicolumn{1}{c|}{$T$ [\si{\degreeCelsius}]} &
		\multicolumn{1}{c|}{$h$ [\si{\kilo\joule\per\kilogram}]} &
		\multicolumn{1}{c|}{$T$ [\si{\degreeCelsius}]} \\
	\hline
	\endhead
	\hline
	\endfoot
		 1 & 10,2000 & 538,00 & 3469,74 & 538,00 & 3469,74 & 538 \\
		 2 &  2,3500 & 345,08 & 3119,02 & 345,08 & 3119,02 & b/d \\
		 3 &  2,3500 & 538,00 & 3548,83 & 538,00 & 3548,83 & 538 \\
		 4 &  0,0035 &  26,67 & 2495,38 &  26,67 & 2495,38 & b/d \\
		 5 &  0,0035 &  26,67 &  111,84 &  26,67 &  111,84 &  33 \\
		 6 &  0,5000 &  26,76 &  112,45 &  26,71 &  112,46 & b/d \\
		 7 &  0,5000 &  84,93 &  356,01 &  84,93 &  356,01 &  84 \\
		 8 &  0,5000 & 115,21 &  483,69 & 115,21 &  483,69 & 116 \\
		 9 &  0,5000 & 151,84 &  640,19 & 151,84 &  640,19 & 146 \\
		10 & 10,2000 & 153,57 &  653,40 & 153,52 &  653,40 & b/d \\
		11 & 10,2000 & 189,04 &  807,53 & 189,04 &  807,52 & 191 \\
		12 & 10,2000 & 220,22 &  946,94 & 220,22 &  946,94 & 231 \\
		u1 &  2,5600 & 354,70 & 3136,47 & 355,25 & 3137,72 & b/d \\
		k1 &  2,5600 & 225,22 &  967,87 & 225,22 &  967,87 & b/d \\
		u2 &  1,3700 & 464,86 & 3398,56 & 480,50 & 3432,51 & b/d \\
		k2 &  1,3700 & 194,04 &  825,61 & 194,04 &  825,61 & b/d \\
		u3 &  0,5000 & 346,95 & 3161,74 & 379,07 & 3228,55 & b/d \\
		u4 &  0,2000 & 258,91 & 2989,18 & 293,45 & 3058,83 & b/d \\
		k4 &  0,2000 & 120,21 &  504,68 & 120,21 &  504,68 & b/d \\
		u5 &  0,0700 & 177,35 & 2832,41 & 203,49 & 2883,72 & b/d \\
		k5 &  0,0700 &  89,93 &  376,68 &  89,93 &  376,68 & b/d \\
\end{longtable}

Na~podstawie zestawionych w~tabeli~\ref{test-wyniki1} zaobserwować
można, iż~program wykonany przez~autora w~większości przypadków wyznacza
identyczne bądź~zbliżone parametry czynnika. Świadczy to o~poprawnej
budowie i~wykorzystaniu modelu czynnika.

Największe różnice wyznaczonych wartości entalpii właściwej występują
dla~upustów turbin oraz~pomp. W~celu wyjaśnienia tych różnic,
skontrolowano również parametry stanu czynnika przy~rozprężaniu
izentropowym ($\eta_i = 1$) i~wyliczono rzeczywiste wartości entalpii
właściwej.

Wartość rzeczywistą dla~turbiny wyznaczono
z~równania~\eqref{test-rown-turbina}, natomiast dla~pompy
z~równania~\eqref{test-rown-pompa}.

\begin{equation}
	\label{test-rown-turbina}
	h = h_i - \eta_i ( h_i - h' )
\end{equation}

\begin{equation}
	\label{test-rown-pompa}
	h = h_i + \frac{ h' - h_i }{ \eta_i }
\end{equation}

Wyznaczone oraz~wyliczone wartości zestawiono
w~tabeli~\ref{test-wyniki2}.

\begin{longtable}{|*{10}{r|}}
	\caption{Zestawienie porównawcze obliczeń rozprężania}
	\label{test-wyniki2} \\

	\hline
		\multirow{2}{*}{Rurociąg} &
		\multicolumn{2}{c|}{Zadane} &
		\multicolumn{2}{c|}{Program} &
		\multicolumn{2}{c|}{Cycle-Tempo} &
		\multicolumn{1}{c|}{Obl.} \\
	\cline{2-8}
		&
		\multicolumn{1}{c|}{$p$ [\si{\mega\pascal}]} &
		\multicolumn{1}{c|}{$h_i$ [\si{\kilo\joule\per\kilogram}]} &
		\multicolumn{1}{c|}{$h'$ [\si{\kilo\joule\per\kilogram}]} &
		\multicolumn{1}{c|}{$h$ [\si{\kilo\joule\per\kilogram}]} &
		\multicolumn{1}{c|}{$h'$ [\si{\kilo\joule\per\kilogram}]} &
		\multicolumn{1}{c|}{$h$ [\si{\kilo\joule\per\kilogram}]} &
		\multicolumn{1}{c|}{$h$ [\si{\kilo\joule\per\kilogram}]} \\
	\hline
	\endfirsthead
	\caption{Zestawienie porównawcze obliczeń rozprężania (c.d.)} \\

	\hline
		\multirow{2}{*}{Rurociąg} &
		\multicolumn{2}{c|}{Zadane} &
		\multicolumn{2}{c|}{Program} &
		\multicolumn{2}{c|}{Cycle-Tempo} &
		\multicolumn{1}{c|}{Obl.} \\
	\cline{2-8}
		&
		\multicolumn{1}{c|}{$p$ [\si{\mega\pascal}]} &
		\multicolumn{1}{c|}{$h_i$ [\si{\kilo\joule\per\kilogram}]} &
		\multicolumn{1}{c|}{$h'$ [\si{\kilo\joule\per\kilogram}]} &
		\multicolumn{1}{c|}{$h$ [\si{\kilo\joule\per\kilogram}]} &
		\multicolumn{1}{c|}{$h'$ [\si{\kilo\joule\per\kilogram}]} &
		\multicolumn{1}{c|}{$h$ [\si{\kilo\joule\per\kilogram}]} &
		\multicolumn{1}{c|}{$h$ [\si{\kilo\joule\per\kilogram}]} \\
	\hline
	\endhead
	\hline
	\endfoot
		\multicolumn{8}{|c|}{\textit{Turbiny}} \\
	\hline
		u1 &  2,5600 & 3469,74 & 3053,15 & 3136,47 & 3053,14 & 3137,72 & 3136,46 \\
		 2 &  2,3500 & 3469,74 & 3031,34 & 3119,02 & 3031,34 & 3119,02 & 3119,02 \\
		u2 &  1,3700 & 3548,83 & 3360,99 & 3398,56 & 3360,99 & 3432,51 & 3398,56 \\
		u3 &  0,5000 & 3548,83 & 3064,97 & 3161,74 & 3064,97 & 3228,55 & 3161,74 \\
		u4 &  0,2000 & 3548,83 & 2849,26 & 2989,18 & 2849,26 & 3058,83 & 2989,17 \\
		u5 &  0,0700 & 3548,83 & 2653,30 & 2832,41 & 2653,30 & 2883,72 & 2832,41 \\
		 4 &  0,0035 & 3548,83 & 2232,02 & 2495,38 & 2232,01 & 2495,38 & 2495,38 \\
	\hline
		\multicolumn{8}{|c|}{\textit{Pompy}} \\
	\hline
		 6 &  0,5000 & 111,836 & 112,329 & 112,454 & 112,33 & 112,46 & 112,452 \\
		10 & 10,2000 & 640,185 & 650,761 & 653,400 & 650,76 & 653,40 & 653,405 \\
\end{longtable}

Obliczenia wykonane przez~program zgodne są w~przybliżeniu
z~obliczeniami wykonanymi na~podstawie zestawionych wyżej wzorów.
Wynika stąd, że~implementacja zależności w~programie jest poprawna.

W~przypadku upustów turbin wyniki dla~rozprężania izentropowego
są~równe, niezgodność występuje natomiast dla~wyników rozprężania
rzeczywistego. Wynika stąd, iż~program Cycle-Tempo wykorzystuje inny
model obliczeń entalpii rozprężania rzeczywistego dla~upustów.

\begin{longtable}{|*{5}{r|}}
	\caption{Zestawienie porównawcze strumieni masy}
	\label{test-wyniki3} \\

	\hline
		\multirow{2}{*}{Rurociąg} &
		\multicolumn{4}{c|}{$\dot m$ [\si{\kilogram\per\second}]} \\
	\cline{2-5}
		&
		\multicolumn{1}{c|}{Program} &
		\multicolumn{1}{c|}{C-T} &
		\multicolumn{1}{c|}{Obl.} &
		\multicolumn{1}{c|}{Lit.} \\
	\hline
	\endfirsthead
	\caption{Zestawienie porównawcze strumieni masy (c.d.)} \\

	\hline
		\multirow{2}{*}{Rurociąg} &
		\multicolumn{4}{c|}{$\dot m$ [\si{\kilogram\per\second}]} \\
	\cline{2-5}
		&
		\multicolumn{1}{c|}{Program} &
		\multicolumn{1}{c|}{C-T} &
		\multicolumn{1}{c|}{Obl.} &
		\multicolumn{1}{c|}{Lit.} \\
	\hline
	\endhead
	\hline
	\endfoot
		 1 & 75,394 & 75,986 & 75,394 & 75,00 \\
		u1 &  4,847 &  4,882 &  4,847 &  5,56 \\
		u2 &  4,248 &  4,226 &  4,249 &  4,56 \\
		u3 &  3,245 &  3,198 &  3,245 &  2,67 \\
		u4 &  3,240 &  3,183 &  3,240 &  3,36 \\
		u5 &  6,085 &  6,024 &  6,085 &  4,42 \\
		wch & 3115,83 & 3157,40 & 3114,31 & b/d \\
\end{longtable}

Z~zestawienia strumieni masy, przedstawionego
w~tabeli~\ref{test-wyniki3} wynika, iż~wartości wyznaczone przez~program
zgodne są z~wartościami obliczonymi na~podstawie schematu cieplnego
obiegu. Różnice w~stosunku do~wartości otrzymanych z~programu
Cycle-Tempo wynikają najprawdopodobniej z~różnic entalpii właściwej
upustów turbin.

\begin{longtable}{|*{4}{r|}}
	\caption{Zestawienie porównawcze mocy}
	\label{test-wyniki4} \\

	\hline
		\multirow{2}{*}{Maszyna} &
		\multicolumn{3}{c|}{$P$ [\si{\kilo\watt}]} \\
	\cline{2-4}
		&
		\multicolumn{1}{c|}{Program} &
		\multicolumn{1}{c|}{C-T} &
		\multicolumn{1}{c|}{Obl.} \\
	\hline
	\endfirsthead
	\caption{Zestawienie porównawcze mocy (c.d.)} \\

	\hline
		\multirow{2}{*}{Maszyna} &
		\multicolumn{3}{c|}{$P$ [\si{\kilo\watt}]} \\
	\cline{2-4}
		&
		\multicolumn{1}{c|}{Program} &
		\multicolumn{1}{c|}{C-T} &
		\multicolumn{1}{c|}{Obl.} \\
	\hline
	\endhead
	\hline
	\endfoot
		generator & 89084,4 & 89060,66 & 00000,00 \\
	\hline
		\multicolumn{4}{|c|}{\textit{Turbiny}} \\
	\hline
		WP & 26283,5 & 26292,92 & 00000,00 \\
		NP & 63831,5 & 63822,08 & 00000,00 \\
	\hline
		\multicolumn{4}{|c|}{\textit{Pompy}} \\
	\hline
		PSk &    38,60 &    40,05 &    00,00 \\
		PR3 &   992,01 &  1014,29 &   000,00 \\
	\hline
		\multicolumn{4}{|c|}{\textit{Kocioł (energia w~paliwie)} } \\
	\hline
		kocioł    & 212872 & 191697,27 & 00000,00 \\
		przegrzew &  33936 &  30561,38 & 00000,00 \\
	\hline
\end{longtable}


\subsection{Obliczenia}

\renewcommand{\arraystretch}{1.5}

\subsubsection{Przykładowe obliczenia entalpii rozprężania/sprężania}

\begin{math}
	\begin{array}{l}
		h_{u1} = h_1 - \eta_i ( h_1 - h_{u1}' ) =
			\num{3469,71} - \num{0,8} \cdot ( \num{3469,71} - \num{3053,15} ) =
			\SI{3136,462}{\kilo\joule\per\kilogram} \\
		h_{u4} = h_3 - \eta_i ( h_3 - h_{u4}' ) =
			\num{3548,83} - \num{0,8} \cdot ( \num{3548,83} - \num{2849,26} ) =
			\SI{2989,174}{\kilo\joule\per\kilogram} \\
		\\
		h_6 = h_5 + \dfrac{h_6' - h_5}{\eta_i} =
			\num{111,836} + \dfrac{\num{112,329} - \num{111,836}}{\num{0,8}} =
			\SI{112,4523}{\kilo\joule\per\kilogram} \\
		h_{10} = h_9 + \dfrac{h_{10}' - h_9}{\eta_i} =
			\num{640,185} + \dfrac{\num{650,761} - \num{640,185}}{\num{0,8}} =
			\SI{653,405}{\kilo\joule\per\kilogram} \\
	\end{array}
\end{math}

\subsubsection{Bilans obiegu cieplnego}

\begin{figure}[H]
	\centering
	\begin{tikzpicture}
		\node[heatxchg,font={\scriptsize}] (R) {R1};

		{ [blue]
			\draw[stealth-] (R.n) -- ++(0,1.5)
				node[pos=.5, left] {$\dot m_{u1}$}
				node[pos=.5, right] {$h_{u1}$};
			\draw[-stealth] (R.se) |- ++(2.75,-1)
				node[pos=.775, above] {$\dot m_{u1}$}
				node[pos=.775, below] {$h_{k1}$};
		}

		{ [green!50!black]
			\draw[stealth-] (R.e) -- ++(2.5,0)
				node[pos=.5, above] {$\dot m$}
				node[pos=.5, below] {$h_{11}$};
			\draw[-stealth] (R.w) -- ++(-2.5,0)
				node[pos=.5, above] {$\dot m$}
				node[pos=.5, below] {$h_{12}$};
		}
	\end{tikzpicture}

	\caption{Otoczenie wymiennika R1}
\end{figure}

\begin{math}
	\begin{array}{l}
		\dot m ( h_{11} - h_{12} ) + \dot m_{u1} ( h_{u1} - h_{k1} ) = 0 \\
		h_{11} - h_{12} + \alpha_1 ( h_{u1} - h_{k1} ) = 0;
			~~~ \alpha_1 = \frac{\dot m_{u1}}{\dot m} \\
		\alpha_1 = \dfrac{h_{12} - h_{11}}{h_{u1} - h_{k1}}
			= \dfrac{\num{946,94} - \num{807,53}}{\num{3136,47} - \num{967,87}}
			= \num{0,06429} \\
	\end{array}
\end{math}

\begin{figure}[H]
	\centering
	\begin{tikzpicture}
		\node[heatxchg,font={\scriptsize}] (R) {R2};

		{ [blue]
			\draw[stealth-] (R.n) -- ++(0,1.5)
				node[pos=.5, left] {$\dot m_{u2}$}
				node[pos=.5, right] {$h_{u2}$};
			\draw[-stealth] (R.se) |- ++(2.75,-1)
				node[pos=.775, above] {$\dot m_{u1} + \dot m_{u2}$}
				node[pos=.775, below] {$h_{k2}$};
			\draw[stealth-] (R.sw) |- ++(-2.75,-1)
				node[pos=.775, above] {$\dot m_{u1}$}
				node[pos=.775, below] {$h_{k1}$};
		}

		{ [green!50!black]
			\draw[stealth-] (R.e) -- ++(2.5,0)
				node[pos=.5, above] {$\dot m$}
				node[pos=.5, below] {$h_{10}$};
			\draw[-stealth] (R.w) -- ++(-2.5,0)
				node[pos=.5, above] {$\dot m$}
				node[pos=.5, below] {$h_{11}$};
		}
	\end{tikzpicture}

	\caption{Otoczenie wymiennika R2}
\end{figure}

\begin{math}
	\begin{array}{l}
		\dot m ( h_{10} - h_{11} ) + \dot m_{u2} h_{u2} + \dot m_{u1} h_{k1}
			- (\dot m_{u1} + \dot m_{u2}) h_{k2} = 0 \\
		\dot m ( h_{10} - h_{11} ) + \dot m_{u1} ( h_{k1} - h_{k2} )
			+ \dot m_{u2} ( h_{u2} - h_{k2} ) = 0 \\
		h_{10} - h_{11} + \alpha_1 ( h_{k1} - h_{k2} )
			+ \alpha_2 ( h_{u2} - h_{k2} ) = 0;
			~~~ \alpha_2 = \frac{\dot m_{u2}}{\dot m} \\

		\alpha_2 = \dfrac{h_{11} - h_{10} - \alpha_1 (h_{k1} - h_{k2})}{
				h_{u2} - h_{k2}} \\
		\alpha_2 = \dfrac{\num{807,53} - \num{653,4} - \num{0,06429} \cdot
				( \num{967,87} - \num{825,61} )}{\num{3398,56} - \num{825,61}}
				= \num{0,05635} \\
	\end{array}
\end{math}

\begin{figure}[H]
	\centering
	\begin{tikzpicture}
		\node[deaerator,font={\scriptsize}] (R) {R3};

		{ [blue]
			\draw[stealth-] (R.n) -- ++(0,1.5)
				node[pos=.5, left] {$\dot m_{u3}$}
				node[pos=.5, right] {$h_{u3}$};
			\draw[-stealth] (R.s) -- ++(0,-1.5)
				node[pos=.5, left] {$\dot m$}
				node[pos=.5, right] {$h_{9}$};
			\draw[stealth-] (R.ne) -- ++(5,0)
				node[pos=.5, above] {$\dot m - \dot m_{u1} - \dot m_{u2} - \dot m_{u3}$}
				node[pos=.5, below] {$h_{8}$};
			\draw[stealth-] (R.nw) -- ++(-5,0)
				node[pos=.5, above] {$\dot m_{u1} + \dot m_{u2}$}
				node[pos=.5, below] {$h_{k2}$};
		}
	\end{tikzpicture}

	\caption{Otoczenie odgazowywacza R3}
\end{figure}

\begin{math}
	\begin{array}{l}
		(\dot m - \dot m_{u1} - \dot m_{u2} - \dot m_{u3}) h_8
			+ (\dot m_{u1} + \dot m_{u2}) h_{k2} + \dot m_{u3} h_{u3}
			- \dot m h_9 = 0 \\
		\dot m ( h_8 - h_9 ) + (\dot m_{u1} + \dot m_{u2})(h_{k2} - h_8)
			+ \dot m_{u3} (h_{u3} - h_8) = 0 \\
		h_8 - h_9 + (\alpha_1 + \alpha_2)(h_{k2} - k_8)
			+ \alpha_3 (h_{u3} - h_8) = 0;
			~~~ \alpha_3 = \frac{\dot m_{u3}}{\dot m} \\

		\alpha_3 = \dfrac{h_9 - h_8 + (\alpha_1 + \alpha_2)(h_8 - h_{k2})
				}{h_{u3} - h_8} \\
		\alpha_3 = \dfrac{\num{640,19} - \num{483,69} + (\num{0,06429}
				+ \num{0,05635}) \cdot (\num{483,69} - \num{825,61})
				}{\num{3161,74} - \num{483,69}} = \num{0,04304} \\
	\end{array}
\end{math}

\begin{figure}[H]
	\centering
	\begin{tikzpicture}
		\node[heatxchg,font={\scriptsize}] (R) {R4};

		{ [blue]
			\draw[stealth-] (R.n) -- ++(0,1.5)
				node[pos=.5, left] {$\dot m_{u4}$}
				node[pos=.5, right] {$h_{u4}$};
			\draw[-stealth] (R.se) |- ++(2.75,-1)
				node[pos=.775, above] {$\dot m_{u4}$}
				node[pos=.775, below] {$h_{k4}$};
		}

		{ [green!50!black]
			\draw[stealth-] (R.e) -- ++(5,0)
				node[pos=.5, above] {$\dot m - \dot m_{u1} - \dot m_{u2} - \dot m_{u3}$}
				node[pos=.5, below] {$h_{7}$};
			\draw[-stealth] (R.w) -- ++(-5,0)
				node[pos=.5, above] {$\dot m - \dot m_{u1} - \dot m_{u2} - \dot m_{u3}$}
				node[pos=.5, below] {$h_{8}$};
		}
	\end{tikzpicture}

	\caption{Otoczenie wymiennika R4}
\end{figure}

\begin{math}
	\begin{array}{l}
		(\dot m - \dot m_{u1} - \dot m_{u2} - \dot m_{u3})
			( h_{7} - h_{8} ) + \dot m_{u4} ( h_{u4} - h_{k4} ) = 0 \\
		(1 - \alpha_1 - \alpha_2 - \alpha_3)( h_7 - h_8 )
			+ \alpha_4 ( h_{u4} - h_{k4} ) = 0 \\

		\alpha_k = 1 - \alpha_1 - \alpha_2 - \alpha_3
			= 1 - \num{0,06429} - \num{0,05635} - \num{0,04304}
			= \num{0,83632} \\
		\alpha_k ( h_7 - h_8 ) + \alpha_4 ( h_{u4} - h_{k4} ) = 0 \\

		\alpha_4 = \dfrac{\alpha_k ( h_8 - h_7 )}{ h_{u4} - h_{k4} }
			= \dfrac{\num{0,83632} \cdot ( \num{483,69} - \num{356,01} )
				}{ \num{2989,18} - \num{504,68} }
			= \num{0,04298} \\
	\end{array}
\end{math}

\begin{figure}[H]
	\centering
	\begin{tikzpicture}
		\node[heatxchg,font={\scriptsize}] (R) {R5};

		{ [blue]
			\draw[stealth-] (R.n) -- ++(0,1.5)
				node[pos=.5, left] {$\dot m_{u5}$}
				node[pos=.5, right] {$h_{u5}$};
			\draw[-stealth] (R.se) |- ++(2.75,-1)
				node[pos=.775, above] {$\dot m_{u4} + \dot m_{u5}$}
				node[pos=.775, below] {$h_{k5}$};
			\draw[stealth-] (R.sw) |- ++(-2.75,-1)
				node[pos=.775, above] {$\dot m_{u4}$}
				node[pos=.775, below] {$h_{k4}$};
		}

		{ [green!50!black]
			\draw[stealth-] (R.e) -- ++(5,0)
				node[pos=.5, above] {$\dot m - \dot m_{u1} - \dot m_{u2} - \dot m_{u3}$}
				node[pos=.5, below] {$h_{6}$};
			\draw[-stealth] (R.w) -- ++(-5,0)
				node[pos=.5, above] {$\dot m - \dot m_{u1} - \dot m_{u2} - \dot m_{u3}$}
				node[pos=.5, below] {$h_{7}$};
		}
	\end{tikzpicture}

	\caption{Otoczenie wymiennika R5}
\end{figure}

\begin{math}
	\begin{array}{l}
		(\dot m - \dot m_{u1} - \dot m_{u2} - \dot m_{u3})( h_{6} - h_{7} )
			+ \dot m_{u5} h_{u5} + \dot m_{u4} h_{k4}
			- (\dot m_{u4} + \dot m_{u5}) h_{k5} = 0 \\
		(\dot m - \dot m_{u1} - \dot m_{u2} - \dot m_{u3})(h_{6} - h_{7})
			+ \dot m_{u4} ( h_{k4} - h_{k5} )
			+ \dot m_{u5} ( h_{u5} - h_{k5} ) = 0 \\
		(1 - \alpha_1 - \alpha_2 - \alpha_3)(h_{6} - h_{7})
			+ \alpha_4 ( h_{k4} - h_{k5} )
			+ \alpha_5 ( h_{u5} - h_{k5} ) = 0 \\
		\alpha_k(h_{6} - h_{7}) + \alpha_4 ( h_{k4} - h_{k5} )
			+ \alpha_5 ( h_{u5} - h_{k5} ) = 0 \\

		\alpha_5 = \dfrac{
			\alpha_k (h_7 - h_6)
			- \alpha_4 ( h_{k4} - h_{k5} )
			}{h_{u5} - h_{k5}} \\
		\alpha_5 = \dfrac{
			\num{0,83632} \cdot (\num{356,01} - \num{112,45})
			- \num{0,04298} \cdot ( \num{504,68} - \num{376,68} )
		}{ \num{2832,41} - \num{376,68} }
			= \num{0,08071} \\
	\end{array}
\end{math}

\begin{figure}[H]
	\centering
	\begin{tikzpicture}
		\node[turbine] (WP) at (0,0) {WP};
		\node[turbine] (NP) at (6,0) {NP};

		{ [red]
			\draw[double] (WP.e) -- (NP.w);
			\draw[double,-stealth] (NP.e) -- ++(2.05,0)
				node[pos=.5, above] {$P_m$};
		}

		% upusty
		\path (NP.se) -- (NP.sw)
			node[pos=.25] (u5) {}
			node[pos=.5] (u4) {}
			node[pos=.75] (u3) {}
			node[pos=1] (u2) {};
		\path (WP.se) -- (WP.sw)
			node[pos=.5] (u1) {};

		{ [blue]
			\draw[stealth-] (WP.nw) -- ++(0,1.5)
				node[pos=.5, left] {$\dot m$}
				node[pos=.5, right] {$h_1$};
			\draw[-stealth] (u1.center) -- ++(0,-.75) -| ++(-.5,-1.5)
				node[pos=.75, left] {$\dot m_{u1}$}
				node[pos=.75, right] {$h_{u1}$};
			\draw[-stealth] (WP.se) -- ++(0,-.55) -| ++(.5,-1.5)
				node[pos=.75, left] {$\dot m_r$}
				node[pos=.75, right] {$h_2$};

			\draw[stealth-] (NP.nw) -- ++(0,1.5)
				node[pos=.5, left] {$\dot m_r$}
				node[pos=.5, right] {$h_3$};
			\draw[-stealth] (u2.center) -- ++(0,-.95) -| ++(-2,-1.5)
				node[pos=.75, left] {$\dot m_{u2}$}
				node[pos=.75, right] {$h_{u2}$};
			\draw[-stealth] (u3.center) -- ++(0,-1.2) -| ++(-1,-1.5)
				node[pos=.85, left] {$\dot m_{u3}$}
				node[pos=.85, right] {$h_{u3}$};
			\draw[-stealth] (u4.center) -- ++(0,-3.5)
				node[pos=.85, left] {$\dot m_{u4}$}
				node[pos=.85, right] {$h_{u4}$};
			\draw[-stealth] (u5.center) -- ++(0,-1) -| ++(1,-1.5)
				node[pos=.85, left] {$\dot m_{u5}$}
				node[pos=.85, right] {$h_{u5}$};
			\draw[-stealth] (NP.se) -- ++(0,-.55) -| ++(2,-1.5)
				node[pos=.75, left] {$\dot m_k$}
				node[pos=.75, right] {$h_4$};
		}

	\end{tikzpicture}

	\caption{Otoczenie turbin}
\end{figure}

\begin{math}
	\begin{array}{l}
		\dot m_r = \dot m - \dot m_{u1} \\
		\dot m_k = \dot m - \sum\limits_{i=1}^5 \dot m_{ui} \\
		\dot m h_1 + \dot m_r (h_3 - h_2) - \dot m_k h_4
			- \sum\limits_{i=1}^5 \dot m_{ui} h_{ui} = \dfrac{P_m}{\eta_m} \\
		\dot m h_1 + (\dot m - \dot m_{u1})(h_3 - h_2)
			- (\dot m - \sum\limits_{i=1}^5 \dot m_{ui}) h_4
			- \sum\limits_{i=1}^5 \dot m_{ui} h_{ui} = \dfrac{P_m}{\eta_m} \\
		\dot m (h_1 + h_3 - h_2 - h_4) + \dot m_{u1} (h_2 - h_3)
			- \sum\limits_{i=1}^5 \dot m_{ui} (h_{ui} - h_{4}) = \dfrac{P_m}{\eta_m} \\

		h_1 + h_3 - h_2 - h_4 + \alpha_1 (h_2 - h_3)
			- \sum\limits_{i=1}^5 \alpha_i (h_{ui} - h_{4}) = \dfrac{P_m}{\dot m \eta_m} \\

		h_1 - h_4 + (1 - \alpha_1)(h_3 - h_2)
			- \sum\limits_{i=1}^5 \alpha_i (h_{ui} - h_{4}) = \dfrac{P_m}{\dot m \eta_m} \\

		\dot m = \dfrac{\frac{P_m}{\eta_m}}{
			h_1 - h_4 + (1 - \alpha_1)(h_3 - h_2)
			- \sum\limits_{i=1}^5 \alpha_i (h_{ui} - h_{4})
		} \\
	\end{array}
\end{math}

\renewcommand{\arraystretch}{1}
\begin{math}
	\begin{array}{lrr}
		~~~~~
		& \num{0,06429} \cdot ( \num{3136,47} - \num{2495,38} ) = & \num{41,2157} \\
		& \num{0,05635} \cdot ( \num{3398,56} - \num{2495,38} ) = & \num{50,8942} \\
		& \num{0,04304} \cdot ( \num{3161,74} - \num{2495,38} ) = & \num{28,6801} \\
		& \num{0,04298} \cdot ( \num{2989,18} - \num{2495,38} ) = & \num{21,2235} \\
		& \num{0,08071} \cdot ( \num{2832,41} - \num{2495,38} ) = & \num{27,2017} \\
		\cline{2-3}
		& \sum\limits_{i=1}^5 \alpha_i (h_{ui} - h_{4}) = & \num{169,2152} \\
	\end{array}
\end{math}

\renewcommand{\arraystretch}{1.5}
\begin{math}
	\begin{array}{l}
		\dot m = \dfrac{\frac{90115}{0,99}}{
			\num{3469,74} - \num{2495,38}
			+ (1 - \num{0,06429}) \cdot (\num{3548,83} - \num{3119,02})
			- \num{169,2152}
		}
			= \SI{75,3943}{\kilogram\per\second}
	\end{array}
\end{math}

\renewcommand{\arraystretch}{1}
\begin{math}
	\left[
		\begin{array}{l}
			\dot m_{u1} \\
			\dot m_{u2} \\
			\dot m_{u3} \\
			\dot m_{u4} \\
			\dot m_{u5} \\
			\dot m_k \\
		\end{array}
	\right] = \left[
		\begin{array}{c}
			\num{0,06429} \\
			\num{0,05635} \\
			\num{0,04304} \\
			\num{0,04298} \\
			\num{0,08071} \\
			\num{0,71263} \\
		\end{array}
	\right] \cdot \num{75,3943} = \left[
		\begin{array}{r}
			\num{4,8471} \\
			\num{4,2485} \\
			\num{3,2450} \\
			\num{3,2404} \\
			\num{6,0851} \\
			\num{53,7282} \\
		\end{array}
	\right] \si{\kilogram\per\second}
\end{math}

\begin{figure}[H]
	\centering
	\begin{tikzpicture}
		\node[condenser] (Sk) at (0,0) {Sk};

		{ [blue]
			\draw[stealth-] (Sk.n) -- ++(0,1.5)
				node[pos=.5, left] {$\dot m_k$}
				node[pos=.5, right] {$h_4$}
				node[pos=1] (tn) {};
			\draw[-stealth] (Sk.s) -- ++(0,-1.5)
				node[pos=.5, left] {$\dot m_k + \dot m_{u1} + \dot m_{u2}$}
				node[pos=.5, right] {$h_5$}
				node[pos=1] (ts) {};
			\draw[stealth-,dashed] (Sk.nw) -- ++(-.25,.25)
				-- ++(-3,0)
				node[pos=.5, above] {$\dot m_{u1} + \dot m_{u2}$}
				node[pos=.5, below] {$h_{k5}$};
		}

		{ [green!50!black]
			\draw[-stealth] (tn) ++(2.25,0) |- (Sk.ne)
				node[pos=.225, left] {$\dot m_{wch}$}
				node[pos=.225, right] {$h_{10°C}$};
			\draw[stealth-] (ts) ++(2.25,0) |- (Sk.se)
				node[pos=.225, left] {$\dot m_{wch}$}
				node[pos=.225, right] {$h_{20°C}$};
		}
	\end{tikzpicture}

	\caption{Otoczenie skraplacza}
\end{figure}

\renewcommand{\arraystretch}{1.5}
\begin{math}
	\begin{array}{l}
		\dot m_k h_4 + (\dot m_{u1} + \dot m_{u2}) h_{k5}
			- (\dot m_k + \dot m_{u1} + \dot m_{u2}) h_5
			- \dot m_{wch} (h_{20°C} - h_{10°C}) = 0 \\
		\dot m_k (h_4 - h_5) + (\dot m_{u1} + \dot m_{u2}) (h_{k5} - h_5)
			- \dot m_{wch} (h_{20°C} - h_{10°C}) = 0 \\

		\dot m_{wch} = \dfrac{\dot m_k (h_4 - h_5)
			+ (\dot m_{u1} + \dot m_{u2})(h_{k5} - h_5)
			}{h_{20°C} - h_{10°C}} \\
		\dot m_{wch} = \dfrac{\num{53,7282} \cdot (\num{2495,38} - \num{111,84})
				+ (\num{4,8471} + \num{4,2485}) \cdot (\num{376,68} - \num{111,84})
				}{\num{84,0118} - \num{42,1174}} \\
		\dot m_{wch} = \SI{3114,31}{\kilogram\per\second} \\
	\end{array}
\end{math}
