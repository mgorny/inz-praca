\section{MODELOWANIE STRUKTURY OBIEGU}

\subsection{Problem}

W~tematyce obiegów cieplnych, najbardziej podstawowym zagadnieniem
obliczeniowym jest bilansowanie. Może ono zostać wykorzystane
do~weryfikacji wcześniej uzyskanych danych bądź~do~wyliczenia
niezbędnych parametrów na~podstawie przyjętych założeń.

Wykonywanie obliczeń bilansowych można najogólniej podzielić na~dwa
etapy:

\begin{enumerate}

\item utworzenie układu równań, opisującego rozpatrywany obieg cieplny;

\item rozwiązanie otrzymanego układu równań.

\end{enumerate}

Drugie z~wymienionych zagadnień można zamknąć ściśle w~domenie
matematyki, dlatego też praca ta nie~będzie wnikała w~nie bardziej
szczegółowo. Znacznie istotniejszy z~punktu widzenia rozpatrywanych
zagadnień jest~sposób tworzenia układu równań.

Rozpatrując bilans układu, w~pierwszej kolejności wyznacza~się
tzw.~osłony bilansowe. Są to umyślne bariery, zakładające, że~wymiana
energii pomiędzy urządzeniem bądź~zespołem urządzeń wyznaczonym osłoną,
a~otoczeniem, następuje jedynie w~ściśle wyznaczony i~mierzalny sposób.

Zastosowanie osłon bilansowych pozwala praktycznie ograniczyć rozważania
do~strumieni wypływających i~wpływających do~układu, względnie
parametrów na~pograniczu i~ogólnych właściwości samego układu. Zaniedbać
można natomiast szczegółowy opis procesów zachodzących wewnątrz
poszczególnych maszyn oraz~przepływy pomiędzy nimi, włączając w~to,
przykładowo, rozpływ czynnika pomiędzy pracujące równolegle skraplacze.

Wybór lokalizacji osłon bilansowych ma znaczny wpływ na~dalszy przebieg
procedury obliczeniowej. Wybór taki powinien~być podyktowany:

\begin{itemize}

\item dostępnymi danymi wejściowymi;

\item poszukiwanymi danymi wyjściowymi;

\item cechami charakterystycznymi i~sposobem połączenia poszczególnych
	maszyn.

\end{itemize}

Przy~założeniu, że~w~bilansie obiegu uwzględnione mają zostać wszystkie
jego elementy, zastosowanie mniejszej liczby osłon bilansowych
(obejmujących jednostkowo większą liczbę urządzeń) zmniejsza liczbę
otrzymanych równań. Jednakże implikuje to~również zmniejszenie liczby
zmiennych, czego skutkiem może~być niemożność uwzględnienia posiadanych
danych bądź~otrzymania poszukiwanych parametrów wprost.

Z~punktu widzenia obliczeń komputerowych w~zakresie objętym tematem
tej~pracy, wielkość układu równań nie~ma znacznego wpływu na~wydajność
programu. Istotne staje~się natomiast uczynienie procesu obliczeniowego
deterministycznym; uwzględnienie większej liczby zmiennych w~układzie
równań pozwala na~poszerzenie funkcjonalności programu bez~konieczności
rozbudowy algorytmu obliczeniowego. Dlatego~też zdecydowano, by~każdy
z~elementów rozpatrywanego układu otoczony~był oddzielną osłoną
bilansową.

Kolejnym krokiem po~ustanowieniu osłon bilansowych jest~zestawienie
równań bilansujących dla~każdego zespołu maszyn, wyznaczonego osłoną.
Równania te~odnoszą~się do~parametrów wewnętrznych zespołu
(jak~na~przykład sprawność) oraz~do~parametrów czynnika
na~wyprowadzeniach.

Najogólniej metody wyznaczania równań podzielić można na~dwie grupy:
metody statyczne oraz~dynamiczne. Metody statyczne charakteryzują~się
tym, że~rozpatrywany układ wyznaczany jest w~fazie projektowania
programu i~zostaje zapisany w~kodzie w~postaci stałych. W~metodach
dynamicznych układ równań tworzony jest podczas pracy programu,
na~podstawie danych wejściowych.


\subsection{Metody statyczne}

Podstawowym założeniem dla~metod statycznych jest niezmienność schematu
rozpatrywanego obiegu cieplnego. Wówczas możliwe staje~się przyjęcie,
że~konstrukcja rozpatrywanego układu równań jest~niezmienna; zmianie
ulegają jedynie wartości zmiennych, względnie współczynników.

Układ równań może zostać wykonany na~etapie projektowania programu
przy~pomocy dowolnych dostępnych metod, a~następnie wprowadzony
w~postaci stałych. Program nie~musi wówczas dysponować logiką
umożliwiającą przetwarzanie schematu obiegu cieplnego; algorytm
ogranicza~się do~przyjęcia danych, rozwiązania układu równań
oraz~wyprowadzenia wyników obliczeń.

Koszt wykonania programu, korzystającego z~metod statycznych,
charakteryzują~się względnie mniejszą składową stałą oraz~składową
zmienną zależną od~złożoności (liczby elementów) rozpatrywanego układu.
Należy jednak wziąć pod~uwagę, że~każdorazowa zmiana schematu wiązać~się
będzie z~kolejnymi kosztami oraz~koniecznością oczekiwania na~ukończenie
modyfikacji kodu programu. W~zasadniczej postaci, metody te
nie~sprawdzą~się w~zastosowaniach, gdzie układ techniczny ulega częstym
zmianom.

Powyższy problem może zostać częściowo rozwiązany poprzez~zastosowanie
wariantu nazywanego ,,metodą zbioru maksimum''. Wówczas układ równań
nie~jest odnoszony do~rzeczywistego obiegu, lecz do~teoretycznego,
celowo przewymiarowanego ,,schematu uniwersalnego''. Poprzez właściwy
dobór parametrów obiegu, możliwe jest skierowanie obliczeniowych
przepływów przez~część z~maszyn i~,,wyłączenie'' pozostałych, tym~samym
uzyskując obieg równoważny rzeczywistemu.

Wykonany w~tej technice program charakteryzuje~się większą
elastycznością. Dobór właściwego schematu uniwersalnego pozwala
na~wykorzystanie go do~obliczeń obiegów, których kształt może~ulegać
zmianom w~krótkich okresach czasu, a~nawet realizację zadań projektowych
w~ograniczonym zakresie. Należy jednak zwrócić, że~większa złożoność
schematu uniwersalnego zwiększa koszty wykonania programu oraz~czyni
jego obsługę trudniejszą.


\subsection{Metody dynamiczne}

W~metodach dynamicznych układ równań tworzony jest w~trakcie pracy
programu. Tym~samym, usuwane są teoretyczne ograniczenia co do~wielkości
obiegu bądź~zastosowanego w~nim układu połączeń. Jednakże jednocześnie
staje~się konieczne zaprojektowanie i~wykonanie modułu umożliwiającego
wprowadzenie schematu obiegu cieplnego oraz~jego analizę.

Istnieją dwie główne metody opisu schematów:

\begin{itemize}
	\item za~pomocą teorii grafów,

	\item za~pomocą macierzy strukturalnych.
\end{itemize}

Obie wymienione metody można określić jako~równoważne. Choć różnią~się
sposobem opisu zagadnień, możliwe jest wierne przeniesienie opisu
z~dowolnie wybranej metody na~metodę alternatywną.

W~zakresie tej~pracy rozważona zostanie jedynie metoda wykorzystująca
teorię grafów. Przy~podejmowaniu tej~decyzji, autor pokierował~się
przede~wszystkim dotychczas posiadaną wiedzą oraz~dostępnością narzędzi,
wykorzystujących teorię grafów.


\subsection{Zastosowanie teorii grafów do~modelowania obiegu}

Podstawowym pojęciem teorii grafów jest~graf. Każdy graf składa~się
ze~zbioru wierzchołków i~zbioru krawędzi. Adaptacja grafu do~modelowania
układu wymaga przypisania wierzchołkom i~krawędziom określonej funkcji.

Istnieją w~tej kwestii dwa rozwiązania:

\begin{enumerate}
	\item modelowanie punktów stanu za~pomocą węzłów, i~przemian
		termodynamicznych za~pomocą krawędzi,

	\item modelowanie punktów stanu za~pomocą krawędzi, i~maszyn
		technicznych za~pomocą węzłów.
\end{enumerate}

\begin{figure}[ht]
	\centering
	\subfloat[graf-przyklad-sch][Schemat cieplny]{
		\begin{tikzpicture}
			{ [thick]
				\node[boiler] (K) at (0,3.5) {K};
				\node[turbine] (T) at (5,4) {T};
				\node[condenser, font=\scriptsize] (S) at (5.75,2) {Sk};
				\node[pumpl, font=\scriptsize] (PS) at (4.5,.5) {PS};
				\node[heatxchg, font=\scriptsize] (R1) at (2.5,.5) {R1};
			}

			{ [font=\scriptsize]
				\node[heatnode] (P0) at (0,5) {};
				\node[above] at (P0) {0};
				\node[heatnode] (P1) at (4.25,5) {};
				\node[above] at (P1) {1};
				\node[heatnode] (P2) at (5.75,3) {};
				\node[right] at (P2) {2};
				\node[heatnode] (P3) at (5.75,1) {};
				\node[left] at (P3) {3};
				\node[heatnode] (P4) at (3.75,.5) {};
				\node[above] at (P4) {4};
				\node[heatnode] (P5) at (1.5,.5) {};
				\node[above] at (P5) {5};
				\node[heatnode] (P6) at (4.25,3) {};
				\node[right] at (P6) {6};
				\node[heatnode] (P7) at (2.5,-.5) {};
				\node[left] at (P7) {7};
				\node[heatnode] (Pi) at (6.5,2.35) {};
				\node[above] at (Pi) {II};
				\node[heatnode] (Pii) at (6.5,1.65) {};
				\node[below] at (Pii) {I};
			}

			{ [-stealth]
				\draw (K.n) |- (2,5) -| (T.nw);
				\draw (T.se) -- (S.n);
				\draw (S.s) |- (PS.e);
				\draw (PS.w) -- (R1.e);
				\draw (R1.w) -| (K.s);

				\draw (T.sw) |- (3,3) -| (R1.n);
				\draw (R1.s) |- (6,-.5) -- ++(0,1.75) -- (S.s);
			}
		\end{tikzpicture}
	} ~%
%
	\subfloat[graf-przyklad-w1][Wariant łukowy maszyn]{
		\begin{tikzpicture}
			{ [font=\scriptsize]
				\node[heatnode] (P0) at (1.5,5) {};
				\node[above] at (P0) {0};
				\node[heatnode] (P1) at (4.25,5) {};
				\node[above] at (P1) {1};
				\node[heatnode] (P2) at (5.75,3) {};
				\node[above] at (P2) {2};
				\node[heatnode] (P3k) at (5.75,.75) {};
				\node[heatnode] (P3) at (5,0) {};
				\node[below] at (P3) {3};
				\node[heatnode] (P4) at (3.5,.75) {};
				\node[left] at (P4) {4};
				\node[heatnode] (P5) at (3.5,3) {};
				\node[above] at (P5) {5};
				\node[heatnode] (P6) at (4.25,3) {};
				\node[above right] at (P6) {6};
				\node[heatnode] (P7) at (4.25,.75) {};
				\node[above right] at (P7) {7};
				\node[heatnode] (Pi) at (6.5,3) {};
				\node[above] at (Pi) {II};
				\node[heatnode] (Pii) at (6.5,.75) {};
				\node[below] at (Pii) {I};
			}

			{ [-stealth, shorten >=1mm, shorten <=1mm, font=\scriptsize]
				\draw (P0) -- (P1) node[above, pos=.5] {(rurociąg)};
				\draw (P1) -- (P6);
				\draw (P6) -- (P2);
				\draw (P2) -- (P3k);
				\draw (P3k) -- (P3);
				\draw (P3) -| (P4) node[above right, pos=.5] {PS};
				\draw (P4) -- (P5);
				\draw (P5) -| (P0) node[above right, pos=.5, font=\small] {K};
				\draw (P6) -- (P7);
				\draw (P7) -- (P3);
				\draw (Pii) -- (Pi);
			}

			{ [color=red, dashdotted, font=\small]
				\draw (P6) +(-.1,-.1) rectangle (5.85,5.1)
					node[below left] {T};
				\draw (P4) +(-.1,-.1) rectangle (4.35,3.1);
				\draw (P3k) +(-.1,-.1) rectangle (6.6,3.1);
				\draw (P7) +(-.1,.1) rectangle (5.85,-.1);
				\path (P7) -- (P3k) node[pos=.5, font=\scriptsize] {(miesz.)};
			}

			{ [color=blue, -stealth, shorten >=1mm, shorten <=1mm, font=\small]
				\draw (P7) ++(0,.75) -- ++(-.75,0);
				\draw (P7) ++(0,1) -- ++(-.75,0);
				\draw (P7) ++(0,1.25) -- ++(-.75,0)
					node[above, pos=.5, color=red] {R1};
				\draw (P3k) ++(0,.75) -- ++(.75,0);
				\draw (P3k) ++(0,1) -- ++(.75,0);
				\draw (P3k) ++(0,1.25) -- ++(.75,0)
					node[above, pos=.5, color=red] {Sk};
			}
		\end{tikzpicture}
	}

	\subfloat[graf-przyklad-w2][Wariant węzłowy maszyn]{
		\begin{tikzpicture}
			{ [thick, rectangle, font=\small, minimum size=8mm]
				\node[draw] (K) at (0,4) {K};
				\node[draw] (ru) at (2,4) {ru};
				\node[draw] (T1) at (4,4) {T$_1$};
				\node[draw] (T2) at (6,4) {T$_2$};
				\node[draw] (S) at (6,2) {Sk};
				\node[draw] (mi) at (6,0) {mi};
				\node[draw] (PS) at (4,0) {PS};
				\node[draw] (R1) at (2,0) {R1};
			}

			{ [-stealth, font=\scriptsize]
				\draw (K.east) -- (ru.west) node[above, pos=.5] {0};
				\draw (ru.east) -- (T1.west) node[above, pos=.5] {1};
				\draw (T1.east) -- (T2.west);
				\draw (T2.south) -- (S.north) node[right, pos=.5] {2};
				\draw (S.south) -- (mi.north);
				\draw (mi.west) -- (PS.east) node[above, pos=.5] {3};
				\draw (PS.west) -- (R1.east) node[above, pos=.5] {4};
				\draw (R1.west) -| (K.south) node[right, pos=.75] {5};

				\draw (T1.south) |- (4,2) -| (R1.north) node[above, pos=.25] {6};
				\draw (R1.south) |- (4,-1) -| (mi.south) node[above, pos=0] {7};

				{ [blue]
					\draw (S.east) ++(0,.25) -- ++(.75,0) node[above, pos=.5] {II};
					\draw (S.east) ++(.75,-.25) -- ++(-.75,0) node[above, pos=.5] {I};
				}
			}
		\end{tikzpicture}
	}

	\caption{Reprezentacja przykładowego obiegu cieplnego za~pomocą teorii grafów}
	\label{metody-teorii-grafow-przyklad}
\end{figure}


Pierwsze z~wyliczonych rozwiązań często traktowano jest jako bardziej
naturalne. Wówczas poszczególne węzły modelują statyczny stan czynnika
w~określonym punkcie układu, a~krawędzie realizują zmianę tego stanu
wskutek przemian termodynamicznych w~maszynach technicznych
i~rurociągach.

Niestety, sytuacji komplikuje~się w~odniesieniu do~maszyn przepływowych,
dla~których rozważany jest więcej niż~jeden obieg czynnika. Przykładem
takich maszyn są~wymienniki ciepła. Modelowanie tego typu maszyny wymaga
zastosowania wielu wierzchołków, a~także wprowadzenia dodatkowej
zależności pomiędzy nimi, która nie~może~być wyrażona wprost za~pomocą
zasadniczego grafu.

Przykładowym rozwiązaniem tego problemu jest~wprowadzenie dodatkowego
grafu, w~którym krawędzie grafu zasadniczego zostaną przekształcone
w~wierzchołki, natomiast wymiana energii pomiędzy odrębnymi obiegami
cieplnymi zostanie wyrażona za~pomocą wierzchołków.

Wadą tego rozwiązania jest~wprowadzenie pośredniej warstwy w~odnoszeniu
schematu fizycznego do~grafu. Choć wierzchołki grafu zasadniczego
mogą~być traktowane jako~wyprowadzenia poszczególnych maszyn,
wyróżnienie maszyn wymaga zarówno uwzględnienia krawędzi grafu
zasadniczego, jak~i~dodatkowych połączeń między nimi w~grafie
pomocniczym.

Rozwiązanie alternatywne stosuje praktycznie odwrotną logikę.
Poszczególne maszyny przepływowe realizowane~są za~pomocą wierzchołków,
zaś~krawędzie wyrażają połączenia pomiędzy nimi. Przemiany
termodynamiczne traktowane~są punktowo, podczas gdy~stan czynnika
zestawiany jest na~łuku.

Rozważany powyżej wymiennik ciepła zestawiony będzie na~grafie za~pomocą
jednego węzła, z~którym połączone będą cztery krawędzie. W~dalszym ciągu
konieczne jest~zestawienie dodatkowych informacji koniecznych
do~rozdzielenia obiegu pierwotnego i~wtórnego, jednakże informacje
te~można zamknąć w~obrębie jednego węzła.

W~tym~przypadku, każda maszyna przepływowa (oraz~rurociąg, jeśli
jest~uwzględniany w~obliczenich) reprezentowana jest na~grafie za~pomocą
dokładnie jednego wierzchołka, zaś~łuki wyrażają wprost przepływ
czynnika (energii) pomiędzy nimi. Dla~wiernego odwzorowania schematu
konieczne~jest uwzględnienie również przepon oddzielających obiegi,
na~przykład poprzez wyznaczenie w~grafie dróg.

Na~rys.~\ref{metody-teorii-grafow-przyklad} zestawiony został prosty
przykład obiegu cieplnego wyrażonego za~pomocą obu wymienionych metod.
W~obu przypadkach uwzględniona została strata na~rurociągu
wyprowadzającym parę świeżą oraz~mieszanie skroplin.
