\section{MODELOWANIE STRUKTURY OBIEGU}

\subsection{Problem}

Pierwszym problemem, jaki powinien zostać rozpatrzony przed~rozpoczęciem
projektowania programu, jest ogólny wybór sposobu modelowanie obiegów.
Od~decyzji tej zależy praktycznie cała struktura programu; co za~tym
idzie, późniejsza zmiana metodyki może wiązać~się z~koniecznością
porzucenia dotychczasowych rezultatów i~rozpoczęcia pracy od~początku.

Najogólniej rzecz biorąc, wybór określonej metodyki implikuje sposób
przenoszenia struktury obiegu, jego parametrów i~charakterystyk
do~postaci cyfrowej, a~w~dalszej części algorytm dokonywania obliczeń.
Z~reguły, metody bardziej złożone (a~co za~tym idzie, bardziej
kosztowne) posiadają mniejsze ograniczenia niż~metody prostsze.


\subsection{Metoda statyczna}

Podstawowym założeniem dla~metody statycznej jest~niezmienność struktury
rozpatrywanego obiegu. Pozwala to na~przeprowadzenie analizy schematu
i~utworzenie układu równań opisującego przemiany jeszcze na~etapie
projektowania programu, a~co za~tym idzie ograniczenia jego logiki
do~rozwiązania stałego układu równań.

Najważniejszą więc zaletą tej~techniki jest~mniejsza złożoność kodu
programu, a~co za~tym idzie krótszy czas i~mniejszy koszt jego
wykonania. Dalsze uproszczenia uzyskać można na~drodze ograniczenia
logiki obliczeniowej, na~przykład poprzez ograniczenie możliwości wyboru
danych i~szukanych.

Należy również podkreślić, że~technika ta pozwala na~całkowicie
indywidualne podejście do~rozpatrywanego układu. Ze~względu na~brak
możliwości wykorzystania programu do~modelowania innych obiegów, jego
kod może zostać w~zupełności dostosowany do~charakterystyk rzeczywistych
urządzeń, i~zoptymalizowany dla~uzyskania największej dokładności
i~wydajności.

Podstawową wadą metody statycznej jest~brak możliwości zmiany
modelowanego układu bez~konieczności wprowadzania znacznych (a~co za~tym
idzie, kosztownych) zmian do~struktury programu. Metoda ta
nie~może zostać zastosowana również, gdy~struktura obiegu nie~jest
jeszcze ustalona, tj.~do~projektowania nowych instalacji.

Podsumowując, metoda ta może znaleźć przede~wszystkim zastosowanie
przy~analizie pracy istniejących układów oraz~rozważaniu nieznacznych
ich modyfikacji; w~sytuacji, kiedy zmiany konstrukcji są na~tyle
rzadkie, iż~koszt wykonania uniwersalnego programu przewyższyłby
przyszłe koszty modernizacji prostszego rozwiązania.

Należy również zwrócić uwagę na~fakt, że~rzeczywista przewaga tej
techniki zależna jest również od~narzędzi dostępnych w~fazie projektowej
(wykorzystywanych do~analizy układu) oraz~złożoności rozpatrywanego
obiektu. Przy~obiegach bardzo złożonych i~braku wyspecjalizowanych
narzędzi, bardziej opłacalna może~być jedna z~metod dynamicznych.


\subsection{Metoda zbioru maksimum}

Rozszerzenie powyższego rozwiązania do~rozwiązywania układów
quasi-dynamicznych możliwe jest poprzez zastosowanie tzw.~,,zbioru
maksimum''. Wówczas, zamiast konkretnego, rzeczywistego obiektu,
w~projekcie tworzy~się obieg przewymiarowany, zawierający znaczną
liczbę nadmiarowych elementów w~różnych konfiguracjach.

Następnie, parametry poszczególnych elementów dobiera~się w~taki sposób,
by~elementy niepotrzebne ,,wyłączyć'', a~z~pozostałych uformować
oczekiwany obieg. Choć w~procedurze obliczeniowej biorą udział wszystkie
elementy, elementy zbędne nie~wpływają na~wynik, charakteryzując~się
chociażby zerowym przepływem.

Zaletą tego rozwiązania jest również względna prostota implementacji
przy~jednoczesnej możliwości modyfikowania kształtu modelowanego obiegu.
Niestety, zmiany ograniczone są ,,z~góry'' pierwotnym kształtem układu,
a~sposób ich wprowadzania może~być uznany za~nieporęczny.

Złożoność obliczeniowa programu jest zależna głównie od~kształtu
podstawowego obiegu. Tym~samym, modelowanie względnie prostych obiektów
może wymagać podobnego nakładu obliczeń jak~obiektów bardzo złożonych.
Podobnie, koszt wykonania programu będzie zbliżony jak dla~programu
operującego na~podstawowym (najbardziej złożonym) układzie.
